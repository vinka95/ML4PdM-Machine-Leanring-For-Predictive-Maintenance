% !TEX root = ../main_sys.tex
%
\chapter{Data Format}
\vspace*{-15mm}\hfill{\fontfamily{phv}\normalsize\emph{Paul Fährmann, Selami Hoxha and Vinay Kaundinya}}
\label{sec:data_format}
\\\\
We define a general data format that can be used for Predictive Maintenance data sets. Our format: \textit{Predictive Maintenance File Format} (PDMFF) is based on the \textit{Attribute Relation File Format} (ARFF\footnote{\href{https://www.cs.waikato.ac.nz/ml/weka/arff.html}{https://www.cs.waikato.ac.nz/ml/weka/arff.html}}) that is commonly used for machine learning data.

\section{Predictive Maintenance File Format (PDMFF)}
\vspace*{-5mm}\hfill{\fontfamily{phv}\normalsize\emph{Paul Fährmann, Vinay Kaundinya}}
\vspace*{+5mm}
\\

PDMFF is is a data format that describes a list of instances for a set of sensors. PDMFF is divided into Header section and Data section. In the header section all the attributes are declared and the data section represents the actual data.
Before we define the keywords in header and data sections, we need to define some basic placeholders.

\textbf{Placeholders}
\begin{itemize}
    \item \textbf{<name>} is a string used to specify a meaningful name for a keyword.
    \item \textbf{<attribute-type>} is used to specify the type of value the \textbf{@ATTRIBUTE} keyword holds.
    \item \textbf{<target-type>} is used to specify the type of value the \textbf{@TARGET} keyword holds.
    \item \textbf{<timestep-type>} is used to specify the data type of timestep. This is used only in TIMESERIES attribute, described later in the chapter.
\end{itemize}

\textbf{Header}: Header section consists metadata of the data that is stored. This information describes different sensors, the type of value recorded relation between the data using keywords such as \textbf{@RELATION}, \textbf{@ATTRIBUTE} and \textbf{@TARGET}.

\textbf{@RELATION} statement defines relation between the different attributes(sensors in our case) by specifying a name.
\begin{verbatim}
   %Format
   @RELATION <name>
   
   %Example
   @RELATION ml4pdm
\end{verbatim}
Each parameter in the dataset is associated with an attribute and is declared using \textbf{@ATTRIBUTE} statement. It also defines a meaning full name using <name> and its associated data type (<attribute-type>).
% Different <attribute-type>s are described in table \ref{table:1}
% \begin{verbatim}
%     @ATTRIBUTE   <name>    <attribute-type>
%  \end{verbatim}
% \begin{table}
%     \caption{Different attribute types}
%     \label{table:1}
%     \begin{tabular}{m{8cm}|m{7cm}}
%         \hline \textbf{Attribute Type}                                                                                                                                    & \textbf{Header and Data Instance } \\ \hline \textbf{NUMERIC} used to specify all numeric parameters. \newline Attribute of this type can take values that are integers, float values etc. \newline \begin{verbatim}@ATTRIBUTE <name> NUMERIC\end{verbatim} &\begin{verbatim}%Header
% @ATTRIBUTE sensor1 NUMERIC

% @DATA 
% %instance 1
% 12.1 \end{verbatim} \\ \hline An attribute can take one of the labels from a set of labels. \newline Attribute is restricted to only the $k$ labels in the set. \newline Format: \begin{verbatim}@ATTRIBUTE <name> {lab_1,..,lab_k}\end{verbatim} & \begin{verbatim}%Header 
% @ATTRIBUTE sensor2 {A,B,C}

% @DATA
% %instance 1
% A \end{verbatim}                                                                                                                                                                        \\ \hline \textbf{DATETIME} is used to specify values of format MM/DD/YYYY-HHMMSSsssss. \newline \begin{verbatim} @ATTRIBUTE <name> DATETIME \end{verbatim} &\begin{verbatim}%Header
% @ATTRIBUTE sensor3 DATETIME

% @DATA
% %instance 1
% 02.12.2004.10.32.39 \end{verbatim}\\ \hline
%         T\textbf{IMESERIES} represents an attribute whose recorded value is timeseries data. \newline Defined using tuples encapsulated within \{<timestep-type>:<attribute-type>\}.\newline <timestep-type> can be of NUMERIC or DATETIME type.\newline
%         \begin{verbatim}
% @ATTRIBUTE <name> TIMESERIES(<timestep-
% type>:<attribute-type>) 
% \end{verbatim}                                                                                                                                            & \begin{verbatim}%Header
% @ATTRIBUTE sensor4 TIMESERIES(NUMERIC:
% NUMERIC)

% @DATA
% %instance 1
% (1:0.0023,2:-0.0027,3:0.0003,..
% ..,31:-0.0006)\end{verbatim}
%         \\ \hline
%         A multidimensional attribute is specified using \textbf{MULTIDIMENSIONAL} keyword.\newline Their dimensions encapsulated in [ ]. \newline \begin{verbatim}@ATTRIBUTE <name> MULTIDIMENSIONAL[dim_1
% dim_2,..,dim_n] \end{verbatim} & \begin{verbatim}%Header
% @ATTRIBUTE sensor5 MULTIDIMENSIONAL[4]

% @DATA
% %instance 1
% [12, 23, 46, 78 ] \end{verbatim}
%         \\ \hline\end{tabular} \end{table}
An <attribute-type> can take following values:
\begin{itemize}
\item NUMERIC used to specify all numeric parameters. Attribute of this type can take values that are integers, float values etc.
\begin{verbatim}
    %Format
    @ATTRIBUTE  <name>  NUMERIC

    %Example
    %Header Instance
    @ATTRIBUTE  sensor1 NUMERIC
    
    @DATA 
    %instance 1
    12.1
\end{verbatim}
% \item STRING used to specify parameters with string values.
% \begin{verbatim}
%         Format --  @ATTRIBUTE   <name>    <attribute-type>
%         ex     --  @ATTRIBUTE   sensor2    STRING     
%     \end{verbatim}
\item An attribute can take one of the labels from a set of labels. Attribute is restricted to only the $k$ labels in the set.
\begin{verbatim}
    %Format
    @ATTRIBUTE  <name>  {lab_1,..,lab_k}  
    
    %Example
    %Header Instance
    @ATTRIBUTE sensor2 {A,B,C}
    
    @DATA
    %instance 1
    A
\end{verbatim}
\item DATETIME is used to specify values of format MM/DD/YYYY-HHMMSSsssss.
\begin{verbatim}
    %Format
    @ATTRIBUTE <name> DATETIME
    
    %Example
    %Header Instance
    @ATTRIBUTE sensor3 DATETIME

    @DATA
    %instance 1
    02.12.2004.10.32.39
\end{verbatim}
\item An attribute whose recorded value is a timeseries can be represented using TIMESERIES attribute-type. A timeseries is defined using tuples encapsulated within { }. Each tuple then has <timestep-type> (data type of time step; NUMERIC, DATETIME) and <attribute-type>(as defined earlier).
\begingroup
\fontsize{10pt}{10pt}\selectfont
\begin{verbatim}
    %Format
    @ATTRIBUTE   <name>    TIMESERIES(<timestep-type>:<attribute-type>)
    
    %Example
    %Header Instance
    @ATTRIBUTE sensor4 TIMESERIES(NUMERIC:NUMERIC)
    
    @DATA
    %instance 1
    (1:0.0023,2:-0.0027,3:0.0003,....,31:-0.0006)
\end{verbatim}
\endgroup
\item An attribute with multidimensional data is specified using MULTIDIMENSIONAL keyword along with their dimensions encapsulated in [ ].
\begingroup
\fontsize{10pt}{12pt}\selectfont
\begin{verbatim}
    %Format
    @ATTRIBUTE   <name>    MULTIDIMENSIONAL[dim_1, dim_2,..,dim_n]
    
    %Example
    %Header Instance
    @ATTRIBUTE sensor5 MULTIDIMENSIONAL[4]

    @DATA
    %instance 1
    [12, 23, 46, 78 ]
\end{verbatim}
\endgroup
\end{itemize}
\textbf{@TARGET} statement is used to specify a target variable and the type of value stored is defined using a <target-type>. Target type can be NUMERIC or value can be chosen from a set of values {value1, value2, value3}.
\begin{verbatim}
    %Format 
    @TARGET  <name>	<target-type>
    
    %Example
    %Header Instance
    @ATTRIBUTE  sensor1  NUMERIC
    @TARGET  class   {A, B, C}

    @DATA
    %instance 1
    12.54#A
 \end{verbatim}
\textbf{Data}: This section of the file contains the actual data recorded at different time instances. Data section begins with @DATA statement. Each instance of data is contained in one row, containing all the attribute values which are delimited by a set of separators.

\textbf{Separators}
\begin{itemize}
    \item $\#$ separates attributes within an instance row.
    \item \textbf{:} separates attributes within a timeseries tuple, {timestep:value}.
    \item \textbf{,} separates timeseries tuples.
    \item \textbf{\%} to comment lines.
\end{itemize}

\section{Examples}
\vspace*{-15mm}\hfill{\fontfamily{phv}\normalsize\emph{Selami Hoxha}}
\label{sec:data_format:examples}

In this section we present two publicly available dataset in
the PDMFF format. Have in mind that the instances separated
by the \# in reality are in one line, but here because of space
limitations are shown in several lines.

\paragraph{Example 1} As the first example we present the Bearing Degradation Simulation
dataset \cite{bearingData} in the PDMFF format. The dataset 2

\begin{verbatim}
        @RELATION bearing_data

        @ATTRIBUTE timestamp DATETIME
        @ATTRIBUTE sensor_1 TIMESERIES(NUMERIC:NUMERIC)
        @ATTRIBUTE sensor_2 TIMESERIES(NUMERIC:NUMERIC)
        @ATTRIBUTE sensor_3 TIMESERIES(NUMERIC:NUMERIC)
        @ATTRIBUTE sensor_4 TIMESERIES(NUMERIC:NUMERIC)

        @DATA
        12/02/2004-103239#
        (1:-0.049,2:-0.042,3:0.015,...,20480:0.020)#
        (1:-0.071,2:-0.073,3:0.000,...,20480:0.076)#
        (1:-0.132,2:-0.007,3:0.007,...,20480:-0.042)#
        (1:-0.010,2:-0.105,3:0.000,...,20480:-0.029)
        02/12/2004-104239#
        (1:-0.088,2:0.022,3:-0.015,...,20480:0.010)#
        (1:-0.127,2:-0.178,3:-0.066,...,20480:0.029)#
        (1:0.154,2:-0.073,3:-0.259,...,20480:-0.112)#
        (1:0.022,2:-0.022,3:-0.056,...,20480:-0.042)
        ...
        02/12/2004-062239#
        (1:--0.002,2:-0.002,3:-0.002,...,20480:-0.002)#
        (1:-0.002,2:-0.002,3:-0.002,...,20480:0.000)#
        (1:-0.002,2:-0.000,3:-0.002,...,20480:-0.000)#
        (1:-0.002,2:-0.002,3:0.000,...,20480:-0.002)
    \end{verbatim}

\paragraph{Example 2}
In this example the CMAPSS dataset \cite{cmapssData} is presented in
the PDMFF format. The presented example is from the test set "test\_FD001" and the RUL
is given in a separate file called "RUL\_FD001" for the test set.

\begin{verbatim}
    @RELATION bearing_data

    @ATTRIBUTE	unit_number             NUMERIC
    @ATTRIBUTE	time_in_cycles          TIMESERIES(NUMERIC:NUMERIC)
    @ATTRIBUTE	operational_setting_1   TIMESERIES(NUMERIC:NUMERIC)
    @ATTRIBUTE	operational_setting_2   TIMESERIES(NUMERIC:NUMERIC)
    @ATTRIBUTE	operational_setting_3   TIMESERIES(NUMERIC:NUMERIC)
    @ATTRIBUTE	sensor_measurement_1    TIMESERIES(NUMERIC:NUMERIC)
    @ATTRIBUTE	sensor_measurement_2    TIMESERIES(NUMERIC:NUMERIC)
    ...
    @ATTRIBUTE	sensor_measurement_21   TIMESERIES(NUMERIC:NUMERIC)
    @TARGET     RUL                     NUMERIC

    @DATA

    %instance 1
    1#
    (1:1,2:2,3:3,...,31:31)# %This could be removed 
    (1:0.0023,2:-0.0027,3:0.0003,...,31:-0.0006)#
    (1:0.0003,2:-0.0003,3:0.0001,...,31:0.0004)#
    ...
    (1:23.3735,2:23.3916,3:23.4166,...,31:23.3552)#
    112
    %instance 2
    2#
    (1:1,2:2,3:3,...,49:49)# %this could be removed
    (1:-0.0009,2:-0.0011,3:0.0002,...,49:0.0018)#
    ...
    (1:-23.3923,2:23.2902,3:23.4064,...,49:23.2618)#
    98
    ...
    %instance 100
    100#
    (1:1,2:2,3:3,...,198:198)# %this could be removed
    (1:0.0014,2:0.0031,3:-0.0000,...,198:0.0013)#
    ...
    (1:23.3087,2:23.3510,3:23.3636,...,198:23.1855)#
    20
\end{verbatim}
