\documentclass[11pt]{meetingmins} % Using meetingmins LaTeX class.

\setcommittee{Project Group: Machine Learning for Predictive Maintenance}

\setdate{March 10, 2021. 09:00 - 12:00}

\setpresent{Tanja Tornede\textit{(Supervisor)}, Christopher Zinda, Paul Fährmann, Sanjay Gupta, Selami Hoxha, Vinay Kaundinya}

\alsopresent{Alexander Tornede\textit{(Guest)}}

\begin{document}

\maketitle

\section{Minutes-taker: Selami Hoxha}

\section{Topics Discussed:}

\subsection{General Class Diagram}
\subsubsection{Details}
\begin{hiddensubitems}
	\item Christopher presented once again the idea on the design of the general class diagram and pointed
	out the changes from last week.
	\item The diagram was described as closer to the goal but there was an issue with the RUL separation
	into Direct RUL and tabular since it would not support every pipeline.
	\item A general remark for rul team given by Tanja was to think of generalizing the use of regressor
	in SVR to use a general regressor.
	\item We discussed the possibility of having HI estimation as a feature extractor in the class diagram
	to make building pipelines easier.
	\item The fit and predict that appears in every class should be removed since they return the same
	thing in every case
	\item Can we have multiple feature extraction with make\_union? The reasoning was that we expect it to
	work since we extend transformer mixin. Alexander suggested that we make a class that has a list of
	feature extraction methods as an array and then implement make\_union.
	\item We consider to handle the error in the pipeline. If the elements in the pipeline don't fit then
	the error will be thrown from the pipeline. We do not use the HI for example anymore in RUL as a class
	but we assume that there was some call that calculates HI in the pipeline.
	\item Alexander suggested to make sure that the pipeline elements are interchangeable, we can think that our framework will be able to be used by three kinds of users: a user that is not interested
	about having everything setup and just put in the dataset to get the result, a intermediate user
	that does not want to create the whole pipeline in its own but still wants to be able to make changes,
	and a user that is able to use our framework elements when creating its own pipeline with
	make\_pipeline for example.
\end{hiddensubitems}
\subsubsection{Aditional Information}
\begin{hiddensubitems}
	\item People involved: Tanja Tornede and Team.
\end{hiddensubitems}

\subsection{General Sequence Diagram}
\subsubsection{Details}
\begin{hiddensubitems}
	\item Make sure that everything is renamed from "approach" to predictor.
	\item We need to include dataset operations that are needed to be able to evaluate the approaches.
	\item In the general sequence without configuration Evaluator parameters should be fixed.
	\item In the sequence diagram with configuration we should make it possible to have a fixed dataset and test different approaches.
	\item If we want the config file to be created per hand, we need to have a clear separation between configuration and serialization. At the moment it seems they are interleaved.
	\item How do we define a pipeline in config file?
	The idea given is to use a string representation of the whole pipeline with
	all the methods and parameters. Another way we could represent the configuration
	file is to use JSON notation.
\end{hiddensubitems}
\subsubsection{Aditional Information}
\begin{hiddensubitems}
	\item People involved: Tanja Tornede and Team.
\end{hiddensubitems}

\subsection{Individual Class Diagrams}
\subsubsection{Details}
\begin{hiddensubitems}
	\item RandomForestApproach in RUL class diagram can be replaced in the tabular case by replacing the regressor with the sckit-learn implementation of sckit-learn.
	\item Also Ensembles are implemented by sckit-learn. The use of this implementation should be
	considered.
	\item Feature extraction team should use the same notation for packages.
	\item Feature extraction team should consider using tensorflow for the implementation for Learning
	with neural networks.
	\item HI class diagram should use packages instead of specific classes from libraries.
\end{hiddensubitems}
\subsubsection{Aditional Information}
\begin{hiddensubitems}
	\item People involved: Tanja Tornede and Team
\end{hiddensubitems}

\subsection{Individual Sequence Diagrams}
\subsubsection{Details}
\begin{hiddensubitems}
	\item The sequence diagrams have all the problem of initializing the class.
	\item The HI sequence diagram should separate the calls for fit and predict and fix the self calls.
\end{hiddensubitems}
\subsubsection{Aditional Information}
\begin{hiddensubitems}
	\item People involved: Tanja Tornede and Team
\end{hiddensubitems}

\section{Next Meeting:}
\begin{hiddensubitems}
	\item Date and time: March 17, 2021 at 09:00 - 11:00
	\item Person responsible for minutes: Vinay Kaundinya
\end{hiddensubitems}

\end{document}