\documentclass[11pt]{meetingmins} % Using meetingmins LaTeX class.

\setcommittee{Project Group: Machine Learning for Predictive Maintenance}

\setdate{February 24, 2020. 09:00 - 11:00}

\setpresent{
  Tanja Tornede\textit{(Supervisor)}, Anurose Prakash, Christopher Zinda,
  Gourav Prakash, Paul Fährmann, Saghar Heidari, Sanjay Gupta, Selami Hoxha,
  Vinay Kaundinya, 
}

\alsopresent{Alexander Tornede\textit{(Guest)}}

\begin{document}

\maketitle

\section{Minutes-taker: Paul Fährmann}

\section{Topics Discussed:}

\subsection{Class Diagrams}
    \subsubsection{Details}
        \begin{hiddensubitems}
            \item Each pipeline-element as well as the whole pipeline needs to be compatible to the scikit  interfaces (BaseEstimator, etc.) and scikit pipeline (\textit{make\_union}, \textit{make\_pipeline})
            \item Create abstract/general diagrams, maybe \textit{Dataset} and \textit{Reader} classes etc.
            \item Check that the naming of classes is consistent over the different diagrams.
            \item The diagrams should be sorted visually.
        \end{hiddensubitems}
	\subsubsection{Aditional Information}
		\begin{hiddensubitems}
			\item People involved: Tanja Tornede, Alexander Tornede and Team
		\end{hiddensubitems}

\subsection{Sequence Diagrams}
	\subsubsection{Details}
		\begin{hiddensubitems}
			\item Create more general \textit{schemas} that are then extended into the different use-cases.
			\item Steps on how the data is transported and the framework is used need diagrams. specifically the steps before \textit{fit/transform} and after \textit{predict}.
			\item Everything that can be called in the program should be visible in sequence diagrams.
		\end{hiddensubitems}
	\subsubsection{Aditional Information}
		\begin{hiddensubitems}
			\item People involved: Tanja Tornede, Alexander Tornede and Team
		\end{hiddensubitems}
\subsection{Data Format}
	\subsubsection{Details}
		\begin{hiddensubitems}
			\item Everything should be put into one file, one line represents one instance.
			\item Discuss time-step types and in general attribute/value-types.
			\item Should research different data-set formats.
			\item Add more simple and short examples for the different value-types.
		\end{hiddensubitems}
	\subsubsection{Aditional Information}
		\begin{hiddensubitems}
			\item People involved: Tanja Tornede and Team
		\end{hiddensubitems}
\subsection{Grading}
	\subsubsection{Details}
		\begin{hiddensubitems}
			\item We get a grade at the end of the next semester (SS2021).
			\item The weights of the milestone grading are hidden.
			\item If we fail at any milestone, we fail the PG.
			\item If Tanja doesn't message you until friday, you passed the first semester of this PG.
		\end{hiddensubitems}
	\subsubsection{Aditional Information}
		\begin{hiddensubitems}
			\item People involved: Tanja Tornede and Team
		\end{hiddensubitems}
\subsection{Planned for next week}
    \subsubsection{Details}
        \begin{hiddensubitems}
            \item Add the architecture document to the ci pipeline. (Christopher Zinda)
            \item Create abstract/general interfaces and class diagrams and put these into the architecture document with short descriptions.
            \item Create a better version of the data format and put it into the architecture document.
        \end{hiddensubitems}
    \subsubsection{Aditional Information}
    	\begin{hiddensubitems}
  		 	\item People involved: Tanja Tornede and Team
  		\end{hiddensubitems}    
    
\section{Next Meeting:}
    \begin{hiddensubitems}
        \item
            Date and time: March 03, 2021 at 09:00 - 11:00
        \item
            Person responsible for minutes: Saghar Heidari
    \end{hiddensubitems}

\end{document}