\documentclass[11pt]{meetingmins} % Using meetingmins LaTeX class.

\setcommittee{Project Group: Machine Learning for Predictive Maintenance}

\setdate{December 9, 2020. 09:00 - 11:00}

\setpresent{
  Tanja Tornede\textit{(Supervisor)}, Anurose Prakash, Christopher Zinda,
  Gourav Prakash, Paul Fährmann, Saghar Heidari, Sanjay Gupta, Selami Hoxha,
  Vinay Kaundinya
}

\absent{}

\alsopresent{}

\begin{document}

\maketitle

\section{Minutes-taker: Saghar Heidari}

\section{Topics Discussed:}

\subsection{Presenting document structure}
    \subsubsection{Details}
        \begin{hiddensubitems}
            \item
              Each team should present document structure in latex format.  
        \end{hiddensubitems}
        
\subsection{Information about (synthetic/real-world) data }
    \subsubsection{Details}
        \begin{hiddensubitems}
            \item
               Three groups (Health state classification, Health index estimation and Remaining useful lifetime) are responsible for searching data that specifically related to their topics.
            \item
               According to type of data (images, temperature data,..),we inform the Feature extraction Team to search specifically in those direction data.
        \end{hiddensubitems}
    \subsubsection{Additional Information}
        \begin{hiddensubitems}
            \item
                Persons Involved: Tanja Tornede and team.
        \end{hiddensubitems}   

\subsection{Information about Taxonomy and presentation of state-of-the-art approaches }
    \subsubsection{Details}
        \begin{hiddensubitems}
            \item
                Document should contain the most common or most promising approaches and it should be generalized.
            \item
                 At the end of this milestone, we will have different approaches in each topic and then choose the most promising or  the most promising combination and then decide together how to implement. 
        \end{hiddensubitems}
        
    \subsubsection{Additional Information}
        \begin{hiddensubitems}
            \item
                Persons Involved: Tanja Tornede and team.
        \end{hiddensubitems}

\subsection{Information about Formal problem definition }
    \subsubsection{Details}
        \begin{hiddensubitems}
            \item
                Two papers were discussed as an example of Formal problem definition and shown how the input and output look like.
        \end{hiddensubitems}
        
    \subsubsection{Additional Information}
        \begin{hiddensubitems}
            \item
                Persons Involved: Tanja Tornede and team.
        \end{hiddensubitems}

\subsection{To do list for next week}
    \subsubsection{Details}
        \begin{hiddensubitems}
            \item
              We need to write Time series formal definition.
             \item
             We need to write clear structure (incl. section titles and bullet points) for each chapter (including Introduction).
             \item We need to assign the sections and subsection (The group makes a suggestion on how to distribute the sections).
             \item
             We need to have a Full group work plan for the milestone 2 (Goal: description on which chapters/sections will be written on, in which weeks and which other tasks will be done at which time).
             
        \end{hiddensubitems}
    
\section{Next Meeting:}
    \begin{hiddensubitems}
        \item
            Date and time: December 16, 2020 at 09:00 - 11:00
        \item
            Person responsible for minutes: Sanjay Gupta
    \end{hiddensubitems}

\end{document}