\documentclass[11pt,a4paper]{article}
\usepackage{graphicx}
\usepackage[utf8]{inputenc}
\usepackage{fancyhdr}
\usepackage{hyperref}
\usepackage[sorting=none, backend=biber]{biblatex}
\usepackage{filecontents}
\begin{filecontents}[overwrite,nosearch]{references.bib}
@article{DBLP:journals/corr/abs-1912-07383,
    author    = {Yongyi Ran and
                 Xin Zhou and
                 Pengfeng Lin and
                 Yonggang Wen and
                 Ruilong Deng},
    title     = {A Survey of Predictive Maintenance: Systems, Purposes and Approaches},
    journal   = {CoRR},
    volume    = {abs/1912.07383},
    year      = {2019},
    url       = {http://arxiv.org/abs/1912.07383},
    archivePrefix = {arXiv},
    eprint    = {1912.07383},
    timestamp = {Wed, 01 Jul 2020 09:22:04 +0200},
    biburl    = {https://dblp.org/rec/journals/corr/abs-1912-07383.bib},
    bibsource = {dblp computer science bibliography, https://dblp.org}
}
@article{DBLP:journals/sj/ZhangYW19,
    author    = {Weiting Zhang and
                Dong Yang and
                Hongchao Wang},
    title     = {Data-Driven Methods for Predictive Maintenance of Industrial Equipment:
                {A} Survey},
    journal   = {{IEEE} Syst. J.},
    volume    = {13},
    number    = {3},
    pages     = {2213--2227},
    year      = {2019},
    url       = {https://doi.org/10.1109/JSYST.2019.2905565},
    doi       = {10.1109/JSYST.2019.2905565},
    timestamp = {Fri, 11 Sep 2020 15:01:32 +0200},
    biburl    = {https://dblp.org/rec/journals/sj/ZhangYW19.bib},
    bibsource = {dblp computer science bibliography, https://dblp.org}
}
@article{DBLP:journals/candie/CarvalhoSVFBA19,
    author    = {Thyago Peres Carvalho and
                Fabr{\'{\i}}zzio Alphonsus A. M. N. Soares and
                Roberto Vita and
                Roberto da Piedade Francisco and
                Jo{\~{a}}o P. Basto and
                Symone G. S. Alcal{\'{a}}},
    title     = {A systematic literature review of machine learning methods applied
                to predictive maintenance},
    journal   = {Comput. Ind. Eng.},
    volume    = {137},
    year      = {2019},
    url       = {https://doi.org/10.1016/j.cie.2019.106024},
    doi       = {10.1016/j.cie.2019.106024},
    timestamp = {Mon, 26 Oct 2020 08:24:22 +0100},
    biburl    = {https://dblp.org/rec/journals/candie/CarvalhoSVFBA19.bib},
    bibsource = {dblp computer science bibliography, https://dblp.org}
}
@phdthesis{DBLP:phd/dnb/Kimotho16,
  author    = {James Kuria Kimotho},
  title     = {Development and performance evaluation of prognostic approaches for
               technical systems},
  school    = {University of Paderborn, Germany},
  year      = {2016},
  url       = {https://nbn-resolving.org/urn:nbn:de:hbz:466:2-27129},
  urn       = {urn:nbn:de:hbz:466:2-27129},
  timestamp = {Wed, 12 Feb 2020 16:42:51 +0100},
  biburl    = {https://dblp.org/rec/phd/dnb/Kimotho16.bib},
  bibsource = {dblp computer science bibliography, https://dblp.org}
}
@article{DBLP:journals/ijon/ChristBNK18,
  author    = {Maximilian Christ and
               Nils Braun and
               Julius Neuffer and
               Andreas W. Kempa{-}Liehr},
  title     = {Time Series FeatuRe Extraction on basis of Scalable Hypothesis tests
               (tsfresh - {A} Python package)},
  journal   = {Neurocomputing},
  volume    = {307},
  pages     = {72--77},
  year      = {2018},
  url       = {https://doi.org/10.1016/j.neucom.2018.03.067},
  doi       = {10.1016/j.neucom.2018.03.067},
  timestamp = {Fri, 31 Jan 2020 14:18:54 +0100},
  biburl    = {https://dblp.org/rec/journals/ijon/ChristBNK18.bib},
  bibsource = {dblp computer science bibliography, https://dblp.org}
}
@article{DBLP:journals/corr/BagnallBLL16,
  author    = {Anthony J. Bagnall and
               Aaron Bostrom and
               James Large and
               Jason Lines},
  title     = {The Great Time Series Classification Bake Off: An Experimental Evaluation
               of Recently Proposed Algorithms. Extended Version},
  journal   = {CoRR},
  volume    = {abs/1602.01711},
  year      = {2016},
  url       = {http://arxiv.org/abs/1602.01711},
  archivePrefix = {arXiv},
  eprint    = {1602.01711},
  timestamp = {Mon, 13 Aug 2018 16:49:02 +0200},
  biburl    = {https://dblp.org/rec/journals/corr/BagnallBLL16.bib},
  bibsource = {dblp computer science bibliography, https://dblp.org}
}
@article{DBLP:journals/corr/abs-1709-01073,
  author    = {Narendhar Gugulothu and
               Vishnu TV and
               Pankaj Malhotra and
               Lovekesh Vig and
               Puneet Agarwal and
               Gautam Shroff},
  title     = {Predicting Remaining Useful Life using Time Series Embeddings based
               on Recurrent Neural Networks},
  journal   = {CoRR},
  volume    = {abs/1709.01073},
  year      = {2017},
  url       = {http://arxiv.org/abs/1709.01073},
  archivePrefix = {arXiv},
  eprint    = {1709.01073},
  timestamp = {Thu, 10 Dec 2020 11:39:55 +0100},
  biburl    = {https://dblp.org/rec/journals/corr/abs-1709-01073.bib},
  bibsource = {dblp computer science bibliography, https://dblp.org}
}
\end{filecontents}
\addbibresource{references.bib}

\title{Project Group:\\
\textbf{Machine Learning for Predictive Maintenance}\\
(Weekly Status Report)}

\author{Supervisors:\\
\textbf{Prof. Dr. Eyke H{\"u}llermeier (eyke@upb.de)}\\
\textbf{Tanja Tornede (tanja.tornede@upb.de)}\\ 
\vspace{10mm}
Name: Sanjay Chhataru Gupta \\
Matriculation Number: 6882964\\
E-mail: sanjaycg@mail.uni-paderborn.de
}
\date{}

% Definition of \maketitle
\makeatletter
\def \@maketitle{
\begin{center}
\includegraphics[width=10cm, height=3cm]{logo.png}\\
{\vspace{12mm} \@title}\\[4ex]
{\@author}\\[4ex]
\end{center}}
\makeatother

\pagestyle{fancy}
\setlength{\headheight}{14pt}
\rhead{PG ML4PdM}
\lhead{Sanjay Gupta}

\begin{document}

\maketitle
\thispagestyle{empty}

\clearpage
\pagenumbering{arabic}

% Summary start.
\newpage
\begin{section}*{Summary of main contributions}
 \begin{subsection}*{Introduction to PdM}
     \begin{enumerate}
         \item Created the latex minutes of meeting (MOM) template with @Vinay.
         \item Created the latex weekly status report template with @Saghar.
         \item Organized doodle to check the availability of all team members to decide the day and timing for the weekly meeting.
         \item Organized doodle to check the availability of all team members to decide day for Presence day.
     \end{enumerate}
 \end{subsection}
 \begin{subsection}*{Topic Study \& Requirement Analysis}
     \begin{enumerate}
         \item Wrote different approaches of PdM part in Introduction chapter.
         \item Wrote Motivation for Time Series Feature extraction (TSFE) chapter.
         \item Wrote the Time Domain (statistical features, BOP, Shapelet Transform, ROCKET, RNN Autoencoder) and Windowing approaches in TSFE chapter.
         \item Reviewed the Introduction, Conclusion, Health State Classification and RUL chapter and added comments and feedback in document.
     \end{enumerate}
 \end{subsection}
 \begin{subsection}*{System Design}
     \begin{enumerate}
         \item Created the sequence diagram for TSFreshWrapper, MovingWeightedAverage, RNNAutoencoder, and WindowingApproach approach and wrote explanation for them in design document.
         \item Wrote Git Strategy chapter in system design document.
         \item Was part of the general class diagrams and sequence diagrams design @Team.
         \item Discussed and created the class diagram for Fixed size feature extractor and Timeseries Transformer With @Paul.
     \end{enumerate}
 \end{subsection}
 \begin{subsection}*{Implementation}
     \begin{enumerate}
         \item Implemented the DatasetSummary, MovingWeightedAverage, TSFreshWrapper, TimeSeriesTransformer, Transformer, EvaluatorConfigParser, and EvaluatorConfig class and also wrote test cases to independently test the functionality of each class.
         \item Created the Anaconda environment setup file for our library and created a README.md file in the GitLab repository to explain how to set up the Anaconda environment on a local machine.
         \item Created Merge request template in GitLab repository.
         \item Reviewed the implementation of RNNAutoencoder class which is implemented by @Christopher.
         \item Added OS path separator while mentioning the file paths in code to avoid OS path separator issues.
         \item Did the code review of Merge request \href{https://git.cs.uni-paderborn.de/machine-learning-for-predictive-maintenance/code/-/merge_requests/44}{44}, \href{https://git.cs.uni-paderborn.de/machine-learning-for-predictive-maintenance/code/-/merge_requests/43}{43}, \href{https://git.cs.uni-paderborn.de/machine-learning-for-predictive-maintenance/code/-/merge_requests/29}{29}, \href{https://git.cs.uni-paderborn.de/machine-learning-for-predictive-maintenance/code/-/merge_requests/27}{27}, \href{https://git.cs.uni-paderborn.de/machine-learning-for-predictive-maintenance/code/-/merge_requests/25}{25}, \href{https://git.cs.uni-paderborn.de/machine-learning-for-predictive-maintenance/code/-/merge_requests/20}{20}, \href{https://git.cs.uni-paderborn.de/machine-learning-for-predictive-maintenance/code/-/merge_requests/19}{19}, \href{https://git.cs.uni-paderborn.de/machine-learning-for-predictive-maintenance/code/-/merge_requests/17}{17}, \href{https://git.cs.uni-paderborn.de/machine-learning-for-predictive-maintenance/code/-/merge_requests/15}{15}.
     \end{enumerate}
 \end{subsection}
 \begin{subsection}*{Final Presentation}
     \begin{enumerate}
         \item Meeting with @Team to discuss and plan presentation contents.
         \item Created the Predictive Maintenance File Format (PdMFF), Feature Engineering(FE), and Abstract Pipeline structure slides and gave presentation on that.
         \item Created API package structure, Authors, License, and Changelog pages in our documentation.
         \item Updated the logo and acknowledgments section in documentation and readme file.
         \item Did review of the Merge request \href{https://git.cs.uni-paderborn.de/machine-learning-for-predictive-maintenance/code/-/merge_requests/49}{49}.
         \item Did research on different types of licenses are available for Open Source Project and suggested BSD 3-clause license for ML4PdM library. Also summarize the license used by dependent package in pdf document.
     \end{enumerate}
 \end{subsection}
\end{section}
% Summary end.

% Calendar Week: 38 page start.
\newpage
\begin{section}*{Calendar Week: 38 \hfill \date{24 September, 2021}}
 \begin{subsection}*{Completed Tasks}
     \begin{enumerate}
         \item Created the first draft of slides to explain \textit{PdMFF} data format.
         \item Created the first draft of slides for Feature Engineering chapter to explain \textit{Timeseries} to \textit{Timeseries} and \textit{Timeseries} to \textit{Fixedsize} transformer.
         \item Discussed few new suggestions for slides and planned few meeting to discussed the new changes.
     \end{enumerate}
 \end{subsection}
 \begin{subsection}*{Tasks in-progress}
     \begin{enumerate}
         \item Updating contents of Feature Engineering slide after @Team feedback.
         \item Create abstract pipeline options slides.
         \item Add \textit{C-MAPSS} dataset introduction slide in presentation.
     \end{enumerate}
 \end{subsection}
\end{section}
% Calendar Week: 38 page end.

% Calendar Week: 37 page start.
\newpage
\begin{section}*{Calendar Week: 37 \hfill \date{17 September, 2021}}
 \begin{subsection}*{Completed Tasks}
     \begin{enumerate}
         \item Completed the \textit{ML4PdM} library structure (hierarchy) display in documentation and also got approval from @Team after demo.
         \item Updated the Authors page in documentation with @Tanja as contact person for future queries related to \textit{ML4PdM} library.
         \item Did research on different types of licenses are available for Open Source Project and suggested suitable license for \textit{ML4PdM} library.
         \item Added BSD 3-clause license to \textit{ML4PdM} library.
         \item Removed contributors section from Authors page in documentation.
         \item Discussed few suggestions and ideas for presentation slides and divided the task between @Team.
     \end{enumerate}
 \end{subsection}
 \begin{subsection}*{Tasks in-progress}
     \begin{enumerate}
         \item Create powerpoint slides for Feature engineering, pdmff data format, few abstract pipeline options.
     \end{enumerate}
 \end{subsection}
\end{section}
% Calendar Week: 37 page end.

% Calendar Week: 36 page start.
\newpage
\begin{section}*{Calendar Week: 36 \hfill \date{10 September, 2021}}
 \begin{subsection}*{Completed Tasks}
     \begin{enumerate}
         \item I was responsible for taking the minutes notes for the week 36 meeting, and wrote the MOM document for the same.
         \item Wrote the Changelogs and Authors page in documentation.
         \item Wrote notebooks examples on How to use \textit{TSFreshWrapper} class of ML4PdM library. So that we can include this examples in documentation.
     \end{enumerate}
 \end{subsection}
 \begin{subsection}*{Tasks in-progress}
     \begin{enumerate}
         \item Working on to display proper tree structure of ML4PdM library in documentation.
         \item Working on the final presentation contents.
     \end{enumerate}
 \end{subsection}
\end{section}
% Calendar Week: 36 page end.

% Calendar Week: 35 page start.
\newpage
\begin{section}*{Calendar Week: 35 \hfill \date{03 September, 2021}}
 \begin{subsection}*{Completed Tasks}
     \begin{enumerate}
         \item Wrote changelog for merge request \href{https://git.cs.uni-paderborn.de/machine-learning-for-predictive-maintenance/code/-/merge_requests/37}{[GEN] Release of version 0.1.0}.
     \end{enumerate}
 \end{subsection}
 \begin{subsection}*{Tasks in-progress}
     \begin{enumerate}
         \item Writing and describing examples for the documentation on how to use \textit{TSFreshWrapper} and \textit{MovingWeightedAverage} class.
     \end{enumerate}
 \end{subsection}
\end{section}
% Calendar Week: 35 page end.

% Calendar Week: 34 page start.
\newpage
\begin{section}*{Calendar Week: 34 \hfill \date{27 August, 2021}}
 \begin{subsection}*{Completed Tasks}
     \begin{enumerate}
         \item Fixed the transform method output representations issues for \textit{MovingWeightedAverage} class.
         \item Implementation of summary print for Dataset class (\href{https://ml4pdm.atlassian.net/browse/ML4PDM-22}{ML4PDM-222}).
         \item Did code review of the Merge requests \href{https://git.cs.uni-paderborn.de/machine-learning-for-predictive-maintenance/code/-/merge_requests/44}{[RUL] Fixes code smell of SVREstimator} and \href{https://git.cs.uni-paderborn.de/machine-learning-for-predictive-maintenance/code/-/merge_requests/43}{[RUL] Implements CNN Rul approach}.
         \item Reviewed the implementation of \textit{RNNAutoencoder} class in feature engineering module.
     \end{enumerate}
 \end{subsection}
 \begin{subsection}*{Tasks in-progress}
     \begin{enumerate}
         \item Working on documentation part for the final project presentation.
     \end{enumerate}
 \end{subsection}
\end{section}
% Calendar Week: 34 page end.

% Calendar Week: 33 page start.
\newpage
\begin{section}*{Calendar Week: 33 \hfill \date{20 August, 2021}}
 \begin{subsection}*{Completed Tasks}
     \begin{enumerate}
         \item Tested the code thoroughly.
         \item Fixed the multiple dataset passing issue in transform method of \textit{TSFreshWrapper} class which was giving error while executing in Pipeline.
         \item Fixed the \textit{TSFreshWrapper} class constructor to assign values to its object.
         \item Refactor the \textit{TSFreshWrapper} and \textit{MovingWeightedAverage} class test code to match with latest code changes.
         \item Updated the docs string of \textit{TSFreshWrapper} and \textit{MovingWeightedAverage} class and its methods.
     \end{enumerate}
 \end{subsection}
 \begin{subsection}*{Tasks in-progress}
     \begin{enumerate}
         \item Fixing the moving weighted average class issue while executing in Pipeline.
         \item Implementation of summary print in Dataset class.
     \end{enumerate}
 \end{subsection}
\end{section}
% Calendar Week: 33 page end.

% Calendar Week: 32 page start.
\newpage
\begin{section}*{Calendar Week: 32 \hfill \date{13 August, 2021}}
 \begin{subsection}*{Completed Tasks}
     \begin{enumerate}
         \item I was responsible for taking the minutes notes for the week 32 meeting, and wrote the MOM document for the same.
         \item Added OS path separator while mentioning the file paths in code to avoid OS path separator issues.
         \item Passing the Dataset object as input and returning Dataset object as output to transform method of \textit{TSFreshWrapper} class.
         \item Implementation of Moving Weighted Average class in Fixed size feature transformation.
         \item Wrote test code to test the Moving Weighted Average class and it's method.
         \item Did code review of the Merge request \href{https://git.cs.uni-paderborn.de/machine-learning-for-predictive-maintenance/code/-/merge_requests/29}{[GEN] Added Integration Test and AttributeFilter} and \href{https://git.cs.uni-paderborn.de/machine-learning-for-predictive-maintenance/code/-/merge_requests/27}{[RUL] Add multiple classifier approach}.
     \end{enumerate}
 \end{subsection}
 \begin{subsection}*{Tasks in-progress}
     \begin{enumerate}
         \item Reviewing the implementation of \textit{RNNAutoencoder} class in feature engineering module.
         \item Testing the code thoroughly.
     \end{enumerate}
 \end{subsection}
\end{section}
% Calendar Week: 32 page end.

% Calendar Week: 31 page start.
\newpage
\begin{section}*{Calendar Week: 31 \hfill \date{06 August, 2021}}
 \begin{subsection}*{Completed Tasks}
     \begin{enumerate}
         \item Completed the passing of list of features as Enum in \textit{TSFreshWrapper} constructor and filtering out feature keys which is not required.
         \item Updated the testing code to test the \textit{TSFreshWrapper} class after latest code changes.
         \item Refactor the \textit{\_pandas\_dataframe\_wrapper} function of \textit{TSFreshWrapper} class to reduce its Cognitive Complexity code smell.
     \end{enumerate}
 \end{subsection}
 \begin{subsection}*{Tasks in-progress}
     \begin{enumerate}
         \item Working on output representations issue of transform method in \textit{TSFreshWrapper} class.
         \item Working on to use OS path separator in code while mentioning the file paths.
         \item Reviewing the implementation of \textit{RNNAutoencoder} class in feature engineering module.
         \item Will start working on my part of pylint code issues.
     \end{enumerate}
 \end{subsection}
\end{section}
% Calendar Week: 31 page end.

% Calendar Week: 30 page start.
\newpage
\begin{section}*{Calendar Week: 30 \hfill \date{30 July, 2021}}
 \begin{subsection}*{Tasks in-progress}
     \begin{enumerate}
         \item Working on to refactor the code of \textit{TSFreshWrapper} class in Feature engineering module based on feedback from @Team.
         \item Working on to use OS path separator in code while mentioning the file paths.
         \item Reviewing the implementation of \textit{RNNAutoencoder} class in feature engineering module.
     \end{enumerate}
 \end{subsection}
\end{section}
% Calendar Week: 30 page end.

% Calendar Week: 29 page start.
\newpage
\begin{section}*{Calendar Week: 29 \hfill \date{23 July, 2021}}
 \begin{subsection}*{Completed Tasks}
     \begin{enumerate}
         \item Fixed the two code smells of the \textit{TSFreshWrapper} class in feature engineering module.
               \begin{enumerate}
                   \item Updated the parameters of \textit{init} method to fix the cognitive complexity code smell.
                   \item Created \textit{TSFreshFeatureCalculatorsAttributes} class with all 72 features as Enum.
                   \item Updated the \textit{TSFreshFeatureCalculators} class with new method which perform the deleting of key from feature calculators dictionary of tsfresh library and also removed \textit{init} method to fix the cognitive complexity code smell.
               \end{enumerate}
     \end{enumerate}
 \end{subsection}
 \begin{subsection}*{Tasks in-progress}
     \begin{enumerate}
         \item Working on to fix the code smell of transform function which is part of \textit{TSFreshWrapper} class in Feature engineering module.
         \item Started reviewing the implementation of \textit{RNNAutoencoder} class in feature engineering module.
     \end{enumerate}
 \end{subsection}
\end{section}
% Calendar Week: 29 page end.

% Calendar Week: 28 page start.
\newpage
\begin{section}*{Calendar Week: 28 \hfill \date{16 July, 2021}}
 \begin{subsection}*{Completed Tasks}
     \begin{enumerate}
         \item I was responsible for taking the minutes notes for the week 28 meeting, and wrote the MOM document for the same.
         \item Did code review of the Merge request \href{https://git.cs.uni-paderborn.de/machine-learning-for-predictive-maintenance/code/-/merge_requests/19}{Added emd signal wrapper and listify function}.
         \item Completed the implementation of \textit{TSFreshWrapper} class and its method in feature engineering module.
               \begin{enumerate}
                   \item Added method to convert list of dataset to pandas dataFrame which is required input format for tsfresh library.
                   \item Added and updated DocString to TSFreshWrapper class and its methods.
                   \item Added FD001\_train pdmff dataset file in test folder to calculate features from tsfresh.
                   \item Added test code to test the implementation of TSFreshWrapper class and its methods.
                   \item Created Merge request in GitLab to review the code from @Paul before mergeing to develop branch.
               \end{enumerate}
     \end{enumerate}
 \end{subsection}
 \begin{subsection}*{Tasks in-progress}
     \begin{enumerate}
         \item Working on the implementation of \textit{RNNAutoencoder} class in feature engineering module.
     \end{enumerate}
 \end{subsection}
\end{section}
% Calendar Week: 28 page end.

% Calendar Week: 27 page start.
\newpage
\begin{section}*{Calendar Week: 27 \hfill \date{09 July, 2021}}
 \begin{subsection}*{Completed Tasks}
     \begin{enumerate}
         \item Implementation of \textit{TSFreshWrapper} class in feature engineering module.
               \begin{enumerate}
                   \item Added all statistical features (64) from \textit{tsfresh} for extraction.
                   \item Added fit and transform method to calculate the features.
                   \item Fixed the circular redundancy issue.
               \end{enumerate}
     \end{enumerate}
 \end{subsection}
 \begin{subsection}*{Tasks in-progress}
     \begin{enumerate}
         \item Writing test module to test the implementation of \textit{TSFreshWrapper} class.
         \item Will start working on the implementation of \textit{RNNAutoencoder} class in feature engineering module.
     \end{enumerate}
 \end{subsection}
\end{section}
% Calendar Week: 27 page end.

% Calendar Week: 26 page start.
\newpage
\begin{section}*{Calendar Week: 26 \hfill \date{02 July, 2021}}
 \begin{subsection}*{Completed Tasks}
     \begin{enumerate}
         \item Implementation of \textit{EvaluatorConfigParser} class.
               \begin{enumerate}
                   \item Moved the \textit{parse} method contents to \textit{parse\_from\_file} method instead of \textit{parse\_from\_string} because it was doing the same task.
                   \item Update the code to use \textit{jsonpickle} instead of \textit{json} library to maintain code consistency.
                   \item Removed the \textit{\_list \, \_data \& list\_of\_} prefix from variable name to reduce the length.
                   \item Moved the iteration of Pipeline paths and strings code to separate static \textit{\_get\_pipelines} method.
                   \item Moved the iteration of Dataset paths code to separate static \textit{\_get\_datasets} method.
                   \item Removed the unwanted code.
                   \item Fixed the code smells.
                   \item Updated the testing code according new changes.
               \end{enumerate}
     \end{enumerate}
 \end{subsection}
 \begin{subsection}*{Tasks in-progress}
     \begin{enumerate}
         \item Implementation of \textit{TSFreshWrapper} class in feature engineering module.
               \begin{enumerate}
                   \item Working on to add \textit{enum} for all features from \textit{tsfresh} library.
               \end{enumerate}
     \end{enumerate}
 \end{subsection}
\end{section}
% Calendar Week: 26 page end.

% Calendar Week: 25 page start.
\newpage
\begin{section}*{Calendar Week: 25 \hfill \date{25 June, 2021}}
 \begin{subsection}*{Completed Tasks}
     \begin{enumerate}
         \item Added \textit{parse} method in \textit {EvaluatorConfigParser} class.
               \begin{enumerate}
                   \item Added iteration of Pipeline paths and string.
                   \item Added iteration of Dataset paths.
                   \item Method return Evaluator object.
                   \item Added Doc string.
               \end{enumerate}
         \item Added test functionality to test the \textit{parse} method of \textit{EvaluatorConfigParser} class.
         \item Updated the \textit{EvaluatorConfig} class parameters with \textit{dataset\_paths \& pipeline\_paths} and its doc string.
         \item Added \textit{EvaluatorConfiguration} and \textit{PipelineConfiguration} json file in test folder so that we can test the \textit{parse} method easily.
     \end{enumerate}
 \end{subsection}
 \begin{subsection}*{Tasks in-progress}
     \begin{enumerate}
         \item Working on the \textit{TSFreshWrapper} class implementation in feature engineering module.
     \end{enumerate}
 \end{subsection}
\end{section}
% Calendar Week: 25 page end.

% Calendar Week: 24 page start.
\newpage
\begin{section}*{Calendar Week: 24 \hfill \date{18 June, 2021}}
 \begin{subsection}*{Tasks in-progress}
     \begin{enumerate}
         \item Working on adding the \textit{parse()} method to the \textit{EvaluatorConfigParser} class, as well as the test functionality which I missed due to my misunderstanding and confusion about the sequence diagram of the pre-defined approach and evaluation.
         \item Fixing few bugs of \textit{EvaluatorConfig} test. i.e.- Instead of passing the database file path, I passed the actual data for testing.
         \item Parallel I'm also working on the \textit{TSFreshWrapper} class's implementation.
     \end{enumerate}
 \end{subsection}
\end{section}
% Calendar Week: 24 page end.

% Calendar Week: 23 page start.
\newpage
\begin{section}*{Calendar Week: 23 \hfill \date{11 June, 2021}}
 \begin{subsection}*{Tasks in-progress}
     \begin{enumerate}
         \item Working on to specify a file path to a pipeline in the config file for an evaluator and it should be
               automatically loaded functionality in \textit{EvaluatorConfigParser} class.
         \item Implementation of \textit{TSFreshWrapper} class.
     \end{enumerate}
 \end{subsection}
\end{section}
% Calendar Week: 23 page end.

% Calendar Week: 22 page start.
\newpage
\begin{section}*{Calendar Week: 22 \hfill \date{04 June, 2021}}
 \begin{subsection}*{Tasks in-progress}
     \begin{enumerate}
         \item Working on to specify a file path to a pipeline in the config file for an evaluator and it should be
               automatically loaded functionality in \textit{EvaluatorConfigParser} class.
         \item Implementation of \textit{TSFreshWrapper} class.
     \end{enumerate}
 \end{subsection}
\end{section}
% Calendar Week: 22 page end.

% Calendar Week: 21 page start.
\newpage
\begin{section}*{Calendar Week: 21 \hfill \date{28 May, 2021}}
 \begin{subsection}*{Completed Tasks}
     \begin{enumerate}
         \item I was responsible for taking the minutes notes for the week 21 meeting, and wrote the MOM document for the same.
         \item Created the \textit{feature/ts-fresh-wrapper} branch which is derived from \textit{develop} branch in GitLab.
     \end{enumerate}
 \end{subsection}
 \begin{subsection}*{Tasks in-progress}
     \begin{enumerate}
         \item Working on to specify a file path to a pipeline in the config file for an evaluator and it should be automatically loaded functionality in \textit{EvaluatorConfigParser} class.
         \item Implementation of \textit{TSFreshWrapper} class.
     \end{enumerate}
 \end{subsection}
\end{section}
% Calendar Week: 21 page end.

% Calendar Week: 20 page start.
\newpage
\begin{section}*{Calendar Week: 20 \hfill \date{21 May, 2021}}
 \begin{subsection}*{Completed Tasks}
     \begin{enumerate}
         \item Fixed the code smells of \textit{Transformer \& TimeSeriesTransformer} abstract class.
         \item Took review for the code changes of \textit{Transformer \& TimeSeriesTransformer} abstract class from @Christopher and @Paul respectively.
         \item Merged the \textit{Transformer \& TimeSeriesTransformer} branch to \textit{develop} branch in GitLab.
     \end{enumerate}
 \end{subsection}
 \begin{subsection}*{Tasks in-progress}
     \begin{enumerate}
         \item Working on to specify a file path to a pipeline in the config file for an evaluator and it should be automatically loaded functionality in \textit{EvaluatorConfigParser} class.
     \end{enumerate}
 \end{subsection}
\end{section}
% Calendar Week: 20 page end.

% Calendar Week: 19 page start.
\newpage
\begin{section}*{Calendar Week: 19 \hfill \date{14 May, 2021}}
 \begin{subsection}*{Completed Tasks}
     \begin{enumerate}
         \item Updated the documentation of \textit{EvaluatorConfig, EvaluatorConfigParser, Transformer \& TimeSeriesTransformer} class to avoid webpage breakdown.
         \item Added the \textit{\_\_init\_\_.py} file in data and parsing folder and adopted to define class name in \textit{\_\_all\_\_} to make importing of class easy in other class or methods.
         \item Updated the \textit{Transformer \& TimeSeriesTransformer} abstract class with python Abstract Base Classes.
         \item Refactor the code to test the \textit{EvaluatorConfig \& EvaluatorConfigParser} class and it's methods based on code reviewers (@Christopher) comments.
     \end{enumerate}
 \end{subsection}
 \begin{subsection}*{Tasks in-progress}
     \begin{enumerate}
         \item Fixing the code smells of \textit{Transformer \& TimeSeriesTransformer} abstract class.
     \end{enumerate}
 \end{subsection}
\end{section}
% Calendar Week: 19 page end.


% Calendar Week: 18 page start.
\newpage
\begin{section}*{Calendar Week: 18 \hfill \date{07 May, 2021}}
 \begin{subsection}*{Completed Tasks}
     \begin{enumerate}
         \item Added the \textit{parse\_from\_string} method in \textit{EvaluatorConfigParser} class and also added the document string in code.
         \item Fixed the recursive call issue of fix and transform methods in \textit{Transformer} abstract class.
         \item Updated the \textit{TimeSeriesTransformer} abstract class and it's methods.
     \end{enumerate}
 \end{subsection}
 \begin{subsection}*{Tasks in-progress}
     \begin{enumerate}
         \item Fixing the code smells of \textit{Transformer \& TimeSeriesTransformer} abstract class.
     \end{enumerate}
 \end{subsection}
\end{section}
% Calendar Week: 18 page end.

% Calendar Week: 17 page start.
\newpage
\begin{section}*{Calendar Week: 17 \hfill \date{30 April, 2021}}
 \begin{subsection}*{Completed Tasks}
     \begin{enumerate}
         \item Implemented the \textit{Transformer} abstract class \& it's methods and also added the class reference to Sphinx documentation for code description rendering on \textit{html} page.
         \item Implemented the \textit{TimeSeriesTransformer} abstract class and also added the class reference to Sphinx documentation for code description rendering on \textit{html} page.
     \end{enumerate}
 \end{subsection}
 \begin{subsection}*{Tasks in-progress}
     \begin{enumerate}
         \item Working on fixing the \textit{EvaluatorConfigParser} class changes as mentioned by @Christopher in merge request review issue.
     \end{enumerate}
 \end{subsection}
\end{section}
% Calendar Week: 17 page end.

% Calendar Week: 16 page start.
\newpage
\begin{section}*{Calendar Week: 16 \hfill \date{23 April, 2021}}
 \begin{subsection}*{Completed Tasks}
     \begin{enumerate}
         \item Created a test module for the \textit{EvaluatorConfigParser} class and tested it with dummy data.
         \item Created a test module for the \textit{EvaluatorConfig} class and tested it with dummy data.
         \item Added the \textit{EvaluatorConfig} class reference to Sphinx documentation for code description rendering on \textit{html} page.
     \end{enumerate}
 \end{subsection}
 \begin{subsection}*{Tasks in-progress}
     \begin{enumerate}
         \item Will start working on implementation of \textit{Transformer} abstract class \href{https://ml4pdm.atlassian.net/browse/ML4PDM-196}{ML4PDM-196}.
     \end{enumerate}
 \end{subsection}
\end{section}
% Calendar Week: 16 page end.

% Calendar Week: 15 page start.
\newpage
\begin{section}*{Calendar Week: 15 \hfill \date{16 April, 2021}}
 \begin{subsection}*{Completed Tasks}
     \begin{enumerate}
         \item I was responsible for taking the minutes notes for the week 15 meeting, and I wrote the MOM document for the same.
         \item Created and implemented the \textit{EvaluatorConfigParser} class and its attributes and methods in \textit{parsing} folder directory of ML4PdM project.
         \item Created and implemented the \textit{EvaluatorConfig} class and its attributes in \textit{data} folder of ML4PdM project.
     \end{enumerate}
 \end{subsection}
 \begin{subsection}*{Tasks in-progress}
     \begin{enumerate}
         \item Implementing and testing the \textit{EvaluatorConfigParser} and \textit{EvaluatorConfig} class and its attributes \& methods.
     \end{enumerate}
 \end{subsection}
\end{section}
% Calendar Week: 15 page end.


% Calendar Week: 14 page start.
\newpage
\begin{section}*{Calendar Week: 14 \hfill \date{09 April, 2021}}
 \begin{subsection}*{Completed Tasks}
     \begin{enumerate}
         \item Updated the \textit{tensorflow} library version from 2.4 to 2.3 in conda environment to resolved the dependencies of \textit{Scipy, tsfresh} libraries.
         \item Set up Visual Studio Code for our project on local machine.
         \item Created \textit{feature/evaluator-config-parser} branch from \textit{develop} branch in GitLab code repository.
     \end{enumerate}
 \end{subsection}
 \begin{subsection}*{Tasks in-progress}
     \begin{enumerate}
         \item Adding and implementing the basic parameters and functions to the EvaluatorConfigParser class. \href{https://ml4pdm.atlassian.net/browse/ML4PDM-192}{ML4PDM-192}
     \end{enumerate}
 \end{subsection}
\end{section}
% Calendar Week: 14 page end.

% Calendar Week: 13 page start.
\newpage
\begin{section}*{Calendar Week: 13 \hfill \date{02 April, 2021}}
 \begin{subsection}*{Completed Tasks}
     \begin{enumerate}
         \item \textit{Python, matplotlib, pandas, scikit-learn, pytest, pywavelets, numpy, scipy, pyts,} and \textit{Keras} libraries installed in the newly created Anaconda environment for the project.
         \item Resolved the \textit{tensorflow} and \textit{tsfresh} library dependencies and added the correct versions to our environment.
         \item Push the final version of the environment.yml file to the GitLab repository so that other members of the team can create and use it.
         \item Created a README.md file in the GitLab repository to explain how to set up the Anaconda environment on a local machine and how to use it.
     \end{enumerate}
 \end{subsection}

 \begin{subsection}*{Tasks in-progress}
     \begin{enumerate}
         \item Adding and implementing the basic parameters and functions to the EvaluatorConfigParser class. \href{https://ml4pdm.atlassian.net/browse/ML4PDM-192}{ML4PDM-192}
         \item Set up Visual Studio Code for our project on local machine.
     \end{enumerate}
 \end{subsection}
\end{section}
% Calendar Week: 13 page end.

% Calendar Week: 12 page start.
\newpage
\begin{section}*{Calendar Week: 12 \hfill \date{26 March, 2021}}
 \begin{subsection}*{Completed Tasks}
     \begin{enumerate}
         \item Removed the DataSet reference from bibliography section of System design document.
         \item Updated the \textit{PytsTransformWrapper, TSFreshWrapper, RNNAutoencoder, and WindowingApproach} approach sequence diagram and their description.
         \item Added footnotes with links in TFE chapter of external libraries which we are going to use in our approach for implementation.
         \item With team decided on Git Strategy and wrote same in Quality Assurance chapter for System Design document.
     \end{enumerate}
 \end{subsection}

 \begin{subsection}*{Tasks in-progress}
     \begin{enumerate}
         \item Working on Anaconda environment setup. \href{https://ml4pdm.atlassian.net/browse/ML4PDM-187}{ML4PDM-187}
     \end{enumerate}
 \end{subsection}
\end{section}
% Calendar Week: 12 page end.

% Calendar Week: 11 page start.
\newpage
\begin{section}*{Calendar Week: 11 \hfill \date{19 March, 2021}}
 \begin{subsection}*{Completed Tasks}
     \begin{enumerate}
         \item Updated the Feature Extraction class diagram boxes alignment \& shape and same added in System design document.
         \item Added Actor element and updated the sequence diagram as per @Tanja comments.
         \item Merge BOP, ROCKET and Shapelet Transform sequence diagrams to one sequence diagram.
         \item Discussed Quality assurance part with team and distributed the work to complete the document.
     \end{enumerate}
 \end{subsection}

 \begin{subsection}*{Tasks in-progress}
     \begin{enumerate}
         \item Updating the System design document for Feature Extraction sequence diagrams.
         \item Working on Git Strategy and it's documentation. \href{https://ml4pdm.atlassian.net/browse/ML4PDM-183}{ML4PDM-183}
     \end{enumerate}
 \end{subsection}
\end{section}
% Calendar Week: 11 page end.

% Calendar Week: 10 page start.
\newpage
\begin{section}*{Calendar Week: 10 \hfill \date{12 March, 2021}}
 \begin{subsection}*{Completed Tasks}
     \begin{enumerate}
         \item Added tsfresh, windowing approach, and moving weighted average class in Feature Extraction class diagram.
         \item Created TswreshWrapper, Shapelet Transform, ROCKET, Bag Of Patterns, RNNAutoencoder and Windowing Approach sequence diagrams and wrote in system design document.
         \item With team, updated the general class and sequence diagram.
     \end{enumerate}
 \end{subsection}

 \begin{subsection}*{Tasks in-progress}
     \begin{enumerate}
         \item Updating the Feature extraction class and sequence diagram as per comments given by @Tanja.
         \item Updating the System design document for Feature Extraction class and sequence diagram.
     \end{enumerate}
 \end{subsection}
\end{section}
% Calendar Week: 10 page end.

% Calendar Week: 09 page start.
\newpage
\begin{section}*{Calendar Week: 09 \hfill \date{05 March, 2021}}
 \begin{subsection}*{Completed Tasks}
     \begin{enumerate}
         \item I was responsible for taking the minutes notes for the week 09 meeting, and I wrote the MOM document for the same.
         \item Meeting with @Christopher, @Paul, @Selami, and @Vinay to refactor the general class diagram, including:
               \begin{itemize}
                   \item Adding the sklearn library classes to the class diagram and extending to the Pipeline element.
                   \item Deciding on the naming convention of class, approaches, and Pipeline.
                   \item Deciding on parser and configuration parser and their data members.
                   \item Deciding on the Evaluation class and its data members and approaches.
                   \item Decide on which approach to adopt for implementation as @Tanja introduced two types of the method definition.
               \end{itemize}
     \end{enumerate}
 \end{subsection}

 \begin{subsection}*{Tasks in-progress}
     \begin{enumerate}
         \item Extending the Feature Extraction part of the class diagram to include all the approaches of the Feature Extraction chapter and determine their data members and functions.
     \end{enumerate}
 \end{subsection}
\end{section}
% Calendar Week: 09 page end.

% Calendar Week: 08 page start.
\newpage
\begin{section}*{Calendar Week: 08 \hfill \date{26 February, 2021}}
 \begin{subsection}*{Completed Tasks}
     \begin{enumerate}
         \item
               Created the class diagram for Feature Extraction chapter on diagrams.net.
               (\href{https://ml4pdm.atlassian.net/browse/ML4PDM-164}{ML4PDM-164})
         \item
               Created the first draft of the system architecture class diagram on diagrams.net with @Anurose, @Christopher, and @Gourav. (\href{https://ml4pdm.atlassian.net/browse/ML4PDM-164}{ML4PDM-164})
         \item
               Created the first draft of the system architecture sequence diagram on diagrams.net with @Anurose, @Christopher, and @Gourav. (\href{https://ml4pdm.atlassian.net/browse/ML4PDM-165}{ML4PDM-165})
     \end{enumerate}
 \end{subsection}

 \begin{subsection}*{Tasks in-progress}
     \begin{enumerate}
         \item
               Working with @Christopher to create refine class and sequence diagrams for the System design of the system. (\href{https://ml4pdm.atlassian.net/browse/ML4PDM-164}{ML4PDM-164}) (\href{https://ml4pdm.atlassian.net/browse/ML4PDM-165}{ML4PDM-165})
     \end{enumerate}
 \end{subsection}
\end{section}
% Calendar Week: 08 page end.

% Calendar Week: 07 page start.
\newpage
\begin{section}*{Calendar Week: 07 \hfill \date{19 February, 2021}}
 \begin{subsection}*{Completed Tasks}
     \begin{enumerate}
         \item
               Reviewed the RUL chapter of Topic study document.
         \item
               Implemented the changes in the TSFE chapter based on the comments of the 2nd review, including:
               \begin{itemize}
                   \item
                         Fixed sentence, spelling mistakes denoted by the reviewers on my part.
                   \item
                         Updated figure font to make it bright.
               \end{itemize}
     \end{enumerate}
 \end{subsection}

 \begin{subsection}*{Tasks in-progress}
     \begin{enumerate}
         \item
               Working with the interface team, e.g. @Anurose, @Christopher, and Gourav, to create class diagrams for the system design milestone.
     \end{enumerate}
 \end{subsection}
\end{section}
% Calendar Week: 07 page end.

% Calendar Week: 06 page start.
\newpage
\begin{section}*{Calendar Week: 06 \hfill \date{12 February, 2021}}
 \begin{subsection}*{Completed Tasks}
     \begin{enumerate}
         \item
               Implemented changes to the PdM approach in the Introduction chapter based on the first review comments, including:
               \begin{itemize}
                   \item
                         Removed the unwanted headings and name which was kept as it is.
                   \item
                         Make provision to add a name to each part of the document section structure so that each team can name it.
                   \item
                         Fixed sentence, spelling mistakes denoted by the reviewers on my part.
               \end{itemize}
         \item
               Implemented the changes in the TSFE chapter based on the comments of the first review, including:
               \begin{itemize}
                   \item
                         Adjust the text of the approach part to fix the image.
                   \item
                         Updated the images to a grid-free background and created a new image to avoid copyright issues.
                   \item
                         Changed the textbf to subsubsection* tag.
                   \item
                         Fixed sentence, spelling mistakes denoted by the reviewers on my part.
                   \item
                         Updated the table and figure latex properties to match the standard decided by the team.
               \end{itemize}
     \end{enumerate}
 \end{subsection}

 \begin{subsection}*{Tasks in-progress}
     \begin{enumerate}
         \item
               Doing the 2nd review of RUL chapter.
     \end{enumerate}
 \end{subsection}
\end{section}
% Calendar Week: 06 page end.

% Calendar Week: 05 page start.
\newpage
\begin{section}*{Calendar Week: 05 \hfill \date{05 February, 2021}}
 \begin{subsection}*{Completed Tasks}
     \begin{enumerate}
         \item
               Reviewed the Health State Classification chapter and added comments, suggestions in the topic study document.
         \item
               Updated the spelling mistakes, date and added the author's name on the Title page of the topic study document. (\href{https://ml4pdm.atlassian.net/browse/ML4PDM-160}{ML4PDM-160})
     \end{enumerate}
 \end{subsection}

 \begin{subsection}*{Tasks in-progress}
     \begin{enumerate}
         \item
               Implementing review provided by reviewers on PdM approaches, time domains, and windowing techniques.
     \end{enumerate}
 \end{subsection}
\end{section}
% Calendar Week: 05 page end.

% Calendar Week: 04 page start.
\newpage
\begin{section}*{Calendar Week: 04 \hfill \date{29 January, 2021}}
 \begin{subsection}*{Completed Tasks}
     \begin{enumerate}
         \item
               Created a template for a Merge request in Gitlab that is reviewed by @Christopher. (\href{https://ml4pdm.atlassian.net/browse/ML4PDM-141}{ML4PDM-141})
         \item
               Reviewed the Introduction, Conclusion chapter and added comments, suggestions in the topic study document.
     \end{enumerate}
 \end{subsection}

 \begin{subsection}*{Tasks in-progress}
     \begin{enumerate}
         \item
               Review of the contents of the Health State chapter written by @Saghar.
     \end{enumerate}
 \end{subsection}
\end{section}
% Calendar Week: 04 page end.

% Calendar Week: 03 page start.
\newpage
\begin{section}*{Calendar Week: 03 \hfill \date{22 January, 2021}}
 \begin{subsection}*{Completed Tasks}
     \begin{enumerate}
         \item
               Wrote RNN Autoencoder method for Time-domain approach in the TSFE chapter.
         \item
               Wrote Window Sliding Method for Windowing Techniques in the TSFE chapter.
         \item
               Wrote the Motivation part for the TSFE chapter.
         \item
               Together with @Christopher, @Paul, and @Vinay, we discussed and decided on Git Branch, merge requests, commit messages, versioning tags, and CI Pipeline for the project.
         \item
               Cross-checked and wrote, PdM Approaches References in the Introduction chapter reference section.
         \item
               Cross checked and wrote, Time domain and windowing techniques references in the TSFE chapter reference section.
     \end{enumerate}
 \end{subsection}

 \begin{subsection}*{Tasks in-progress}
     \begin{enumerate}
         \item
               Reviewing the contents which I wrote in the Topic study document.
     \end{enumerate}
 \end{subsection}
\end{section}
% Calendar Week: 03 page end.

% Calendar Week: 02 page start.
\newpage
\begin{section}*{Calendar Week: 02 \hfill \date{15 January, 2021}}
 \begin{subsection}*{Completed Tasks}
     \begin{enumerate}
         \item
               Added a description to each of the basic mathematical features of the time domain as per @Tanja feedback.
         \item
               Updated the Overview of fault diagnosis and Bags of Patterns images as per @Tanja feedback.
         \item
               With @Vinay, discuss the windowing techniques required by the RUL team to be included in the TSFE chapter.
         \item
               Wrote ROCKET method for Time domain approach in TSFE chapter.
     \end{enumerate}
 \end{subsection}

 \begin{subsection}*{Tasks in-progress}
     \begin{enumerate}
         \item
               Writing Autoencoder method for Time domain approach in TSFE chapter.
     \end{enumerate}
 \end{subsection}
\end{section}
% Calendar Week: 02 page end.

% Calendar Week: 01 page start.
\newpage
\begin{section}*{Calendar Week: 01 \hfill \date{08 January, 2021}}
 \begin{subsection}*{Completed Tasks}
     \begin{enumerate}
         \item
               Wrote basic mathematical features for Time domain approach in TSFE chapter.
         \item
               Wrote Bags-of-patterns method for Time domain approach in TSFE chapter.
         \item
               Wrote Shapelet Transform method for Time domain approach in TSFE chapter.
     \end{enumerate}
 \end{subsection}

 \begin{subsection}*{Tasks in-progress}
     \begin{enumerate}
         \item
               Writing ROCKET method for Time domain approach in TSFE chapter.
     \end{enumerate}
 \end{subsection}
\end{section}
% Calendar Week: 01 page end.

% Calendar Week: 52-53 page start.
\newpage
\begin{section}*{Calendar Week: 52-53 \hfill \date{30 December, 2020}}
 \begin{subsection}*{Completed Tasks}
     \begin{enumerate}
         \item
               With @Paul, discussed the new structure and added few topics to our Times series feture extraction (TSFE) chapter.
         \item
               Wrote the different PdM approaches in the topic study document under Introduction chapter.
         \item
               Wrote a brief explanation for the TSFE approach section.
     \end{enumerate}
 \end{subsection}

 \begin{subsection}*{Challenges}
     \begin{enumerate}
         \item
               Writing mathematical equations and representations in LaTex.
     \end{enumerate}
 \end{subsection}

 \begin{subsection}*{Tasks in-progress}
     \begin{enumerate}
         \item
               Writing time domain feature extraction methods in TSFE chapter.
     \end{enumerate}
 \end{subsection}
\end{section}
% Calendar Week: 52-53 page end.

% Calendar Week: 51 page start.
\newpage
\begin{section}*{Calendar Week: 51 \hfill \date{18 December, 2020}}
 \begin{refsection}
     \begin{subsection}*{Completed Tasks}
         \begin{enumerate}
             \item
                   Responsible for taking the notes and writing of the Minutes of meeting report for week 51.
             \item
                   Worked with the team to refine the formal definition of time series.
             \item
                   Finished reading of the seed papers \cite{DBLP:journals/ijon/ChristBNK18}, \cite{DBLP:journals/corr/BagnallBLL16} of Time Series Feature Extraction.
             \item
                   With @Tanja and @Paul, the contents of the Time Series Feature Extraction chapter are discussed.
             \item
                   Updated the Milestone 2 work plan for our topic with @Paul and sent the work plan to @Vinay to create the Gantt chart and represent to @Tanja.
         \end{enumerate}
     \end{subsection}

     \begin{subsection}*{Tasks in-progress}
         \begin{enumerate}
             \item
                   Writing the different Approaches of PdM in Introduction chapter of topic study document.
         \end{enumerate}
     \end{subsection}

     \printbibliography
 \end{refsection}
\end{section}
% Calendar Week: 51 page end.

% Calendar Week: 50 page start.
\newpage
\begin{section}*{Calendar Week: 50 \hfill \date{11 December, 2020}}
 \begin{refsection}
     \begin{subsection}*{Completed Tasks}
         \begin{enumerate}
             \item
                   Worked with the team to decide on Introduction chapter for the topic study.
             \item
                   Finished reading the seed paper \cite{DBLP:journals/corr/abs-1709-01073} of Time Series Feature Extraction. (\href{https://ml4pdm.atlassian.net/browse/ML4PDM-81}{ML4PDM-81})
             \item
                   With @Paul, discussed about seed papers \cite{DBLP:journals/ijon/ChristBNK18}, \cite{DBLP:journals/corr/BagnallBLL16}, \cite{DBLP:journals/corr/abs-1709-01073} and dissertation \cite{DBLP:phd/dnb/Kimotho16} contents related to time series and their methods for PdM.
             \item
                   With @Paul, planned Times Series Feature Extraction chapter outlines, who is going to work on which sub sections and deadline to finished that part.
         \end{enumerate}
     \end{subsection}

     \begin{subsection}*{Tasks in-progress}
         \begin{enumerate}
             \item
                   Reading the seed papers \cite{DBLP:journals/ijon/ChristBNK18}, \cite{DBLP:journals/corr/BagnallBLL16} of Time Series Feature Extraction.
             \item
                   Writing and research on the Different Approaches of PdM in Introduction chapter of topic study document.
         \end{enumerate}
     \end{subsection}

     \printbibliography
 \end{refsection}
\end{section}
% Calendar Week: 50 page end.

% Calendar Week: 49 page start.
\newpage
\begin{section}*{Calendar Week: 49 \hfill \date{04 December, 2020}}
 \begin{refsection}
     \begin{subsection}*{Completed Tasks}
         \begin{enumerate}
             \item
                   Worked with the team to decide on a template and its structure for the topic study.
             \item
                   With @Paul, planned how we're going to work on the topic "Times Series Feature Extraction" (TSFE).
         \end{enumerate}
     \end{subsection}

     \begin{subsection}*{Challenges}
         \begin{enumerate}
             \item
                   Understand how we need to modify the TSFE chapter outline, as this topic will also be used for reference by all other team members in the future.
             \item
                   Trying to understand different types of feature extraction methods and their purposes.
         \end{enumerate}
     \end{subsection}

     \begin{subsection}*{Tasks in-progress}
         \begin{enumerate}
             \item
                   Reading the dissertation \cite{DBLP:phd/dnb/Kimotho16} and the third paper \cite{DBLP:journals/corr/abs-1709-01073} listed on the ”Milestone 2: Topics” document for our topic Times Series Feature Extraction.
         \end{enumerate}
     \end{subsection}

     \printbibliography
 \end{refsection}
\end{section}
% Calendar Week: 49 page end.

% Calendar Week: 48 page start.
\newpage
\begin{section}*{Calendar Week: 48 \hfill \date{27 November, 2020}}
 \begin{refsection}
     \begin{subsection}*{Completed Tasks}
         \begin{enumerate}
             \item
                   Finished the reading of survey paper \cite{DBLP:journals/candie/CarvalhoSVFBA19}. (\href{https://ml4pdm.atlassian.net/browse/ML4PDM-30}{ML4PDM-30})
                   \par In this paper, the authors systematically reviewed the literature on PdM and its technique. In the first section, the types of maintenance strategies used in industry are introduced. Subsequently, in section two different publications and journals are considered for analysis and selection of the most related PdM papers.In addition, the Citation Analysis Platform is also introduced to help researchers check the number of citations for selected papers.In the final review of the literature, Machine Learning and Deep Learning Techniques are discussed briefly.
         \end{enumerate}
     \end{subsection}

     \begin{subsection}*{Tasks in-progress}
         \begin{enumerate}
             \item
                   Reading the dissertation \cite{DBLP:phd/dnb/Kimotho16}.
         \end{enumerate}
     \end{subsection}

     \printbibliography
 \end{refsection}
\end{section}
% Calendar Week: 48 page end.

% Calendar Week: 47 page start.
\newpage
\begin{section}*{Calendar Week: 47 \hfill \date{20 November, 2020}}
 \begin{refsection}
     \begin{subsection}*{Completed Tasks}
         \begin{enumerate}
             \item
                   Finished the reading of survey paper \cite{DBLP:journals/corr/abs-1912-07383}. (\href{https://ml4pdm.atlassian.net/browse/ML4PDM-14}{ML4PDM-14})
                   \par The authors of this paper provided a comprehensive analysis of past and current literature on the design, purposes and methods of the PdM. The three different machine architecture of PdM is explained in detail in the first section. Then the objectives of PdM in cost minimization, availability and realiablilty are determined in the next section. Various ML and DL techniques describe the achievement of the PdM in detail in the last section.
             \item
                   Finished the reading of survey paper \cite{DBLP:journals/sj/ZhangYW19}. (\href{https://ml4pdm.atlassian.net/browse/ML4PDM-33}{ML4PDM-33})
                   \par Compared to the research survey paper \cite{DBLP:journals/corr/abs-1912-07383}, survey paper \cite{DBLP:journals/sj/ZhangYW19} is more inclined towards industrial equipment. It began with a brief introduction to the PdM, the intent of the PdM, and data-driven methods. In this paper, the authors examine the industrial applications of the last five years and how the industry is trying to use PdM in an accurate and efficient manner. The challenges, advantages, and disadvantages of the different ML and DL application scenarios are also described in detail.
         \end{enumerate}
     \end{subsection}

     \begin{subsection}*{Tasks in-progress}
         \begin{enumerate}
             \item
                   Reading the survey paper \cite{DBLP:journals/candie/CarvalhoSVFBA19}. (\href{https://ml4pdm.atlassian.net/browse/ML4PDM-30}{ML4PDM-30})
         \end{enumerate}
     \end{subsection}

     \printbibliography
 \end{refsection}
\end{section}
% Calendar Week: 47 page end.

% Calendar Week: 46 page start.
\newpage
\begin{section}*{Calendar Week: 46 \hfill \date{13 November, 2020}}
 \begin{refsection}
     \begin{subsection}*{Completed Tasks}
         \begin{enumerate}
             \item
                   Added Minutes-Taker section and person responsible to take the minutes of the next meeting in the Overleaf minutes of meeting template. After that, I uploaded the template to GitLab so that other members of the team could use it as a reference. (\href{https://ml4pdm.atlassian.net/browse/ML4PDM-38}{ML4PDM-38})
             \item
                   A reference section has been added to each week page so that each week has its own reference section in the weekly status report and the 'References.bib' file has been deleted. After that, I uploaded the template to GitLab so that other members of the team could use it as a reference. (\href{https://ml4pdm.atlassian.net/browse/ML4PDM-37}{ML4PDM-37})
             \item
                   Completed the reading and understanding of abstract, Introduction, Categories of Maintenance Techniques, and System architecture of PdM section of survey paper \cite{DBLP:journals/corr/abs-1912-07383}.
             \item
                   Set up the GitLab environment on the local machine.
         \end{enumerate}
     \end{subsection}

     \begin{subsection}*{Tasks in-progress}
         \begin{enumerate}
             \item
                   Started reading the Purposes of PdM section of survey paper \cite{DBLP:journals/corr/abs-1912-07383}.
         \end{enumerate}
     \end{subsection}

     \printbibliography
 \end{refsection}
\end{section}
% Calendar Week: 46 page end.

% Calendar Week: 45 page start.
\newpage
\begin{section}*{Calendar Week: 45 \hfill \date{06 November, 2020}}
 \begin{refsection}
     \begin{subsection}*{Completed Tasks}
         \begin{enumerate}
             \item
                   Organize the Doodle Poll to check the availability of all team members to decide the day \&\ the timing for weekly meeting with the supervisors. (\href{https://doodle.com/poll/3u7z6hh52wvttfhz?utm_source=poll&utm_medium=link}{ML4PdM - Weekly Meeting})
             \item
                   Organize the Doodle Poll to check the availability of all team members to decide on the day of the Presence Day. (\href{https://doodle.com/poll/2p2dmvq7xgvkmgf7?utm_source=poll&utm_medium=link}{ML4PdM - Presence Day})
             \item
                   Along with Saghar Heidari, we created the Weekly status report template in Overleaf as decided by supervisors and team. After that, I uploaded the template to GitLab so that other team members could use it as a reference. (\href{https://ml4pdm.atlassian.net/browse/ML4PDM-37}{ML4PDM-37})
             \item
                   Along with Vinay Kaundinya, we created the Minutes of meeting template in Overleaf as decided by supervisors and team. After that, I uploaded the template to GitLab so that other team members could use it as a reference. (\href{https://ml4pdm.atlassian.net/browse/ML4PDM-38}{ML4PDM-38})
         \end{enumerate}
     \end{subsection}

     \begin{subsection}*{Tasks in-progress}
         \begin{enumerate}
             \item
                   Started reading the survey paper \cite{DBLP:journals/corr/abs-1912-07383} to expand my knowledge of Machine Learning for Predictive Maintenance.
         \end{enumerate}
     \end{subsection}

     \printbibliography
 \end{refsection}
\end{section}
% Calendar Week: 45 page end.

\end{document}