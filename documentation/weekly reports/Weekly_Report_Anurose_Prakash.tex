\documentclass[11pt,a4paper]{article}
\usepackage{graphicx}
\usepackage[utf8]{inputenc}
\usepackage{fancyhdr}
\usepackage{hyperref}

\title{Project Group:\\
\textbf{Machine Learning for Predictive Maintenance}\\
(Weekly Status Report)}

\author{Supervisors:\\
\textbf{Prof. Dr. Eyke H{\"u}llermeier (eyke@upb.de)}\\
\textbf{Tanja Tornede (tanja.tornede@upb.de)}\\ 
\vspace{10mm}
Name: Anurose Prakash \\
Matriculation Number: 6905173\\
E-mail: anurosep@mail.uni-paderborn.de
}
\date{}

% Definition of \maketitle
\makeatletter
\def \@maketitle{
\begin{center}
\includegraphics[width=10cm, height=3cm]{logo.png}\\
{\vspace{12mm} \@title}\\[4ex]
{\@author}\\[4ex]
\end{center}}
\makeatother

\pagestyle{fancy}
\setlength{\headheight}{14pt}
\rhead{PG ML4PdM}
\lhead{Anurose Prakash}

\begin{document}

\maketitle
\thispagestyle{empty}

\clearpage
\pagenumbering{arabic}

%If u want to add more pages copy the below code and paste it here.
\newpage
\begin{section}*{Calendar Week: 7 \hfill \date{19 February, 2021}}
    \begin{subsection}*{Completed Tasks}
        \begin{enumerate}
            \item 
            Updated the health state classification chapter with second review comments of Paul.            
        \end{enumerate}
    \end{subsection}
    
    \begin{subsection}*{Tasks in-progress}
        \begin{enumerate}
            \item
                Working with interface team i.e. Christopher, Sanjay and Gourav for creating class diagrams for the system design milestone.
            \item 
                Checking with Saghar on the contents for UML diagrams specific to health state classification chapter. 
        \end{enumerate}
    \end{subsection}
\end{section}

\newpage
\begin{section}*{Calendar Week: 6 \hfill \date{11 February, 2021}}
    \begin{subsection}*{Completed Tasks}
        \begin{enumerate}
            \item 
                Updated the alpha version of health state classification chapter with comments given by Selami and Sanjay.
            \item
                Reviewed the feature extraction chapter and prepared a feedback for that chapter.
        \end{enumerate}
    \end{subsection}
    
    \begin{subsection}*{Tasks in-progress}
        \begin{enumerate}
            \item
                Need to update the health state classification chapter with second review comments of Paul and Gourav.
        \end{enumerate}
    \end{subsection}
\end{section}

\newpage
\begin{section}*{Calendar Week: 5 \hfill \date{5 February, 2021}}
    \begin{subsection}*{Completed Tasks}
        \begin{enumerate}
            \item
                Reviewed the health index estimation chapter and prepared a feedback for that chapter.
        \end{enumerate}
    \end{subsection}
    
    \begin{subsection}*{Tasks in-progress}
        \begin{enumerate}
            \item
                Working on review comments of the health state classification chapter given by Selami and Sanjay.
        \end{enumerate}
    \end{subsection}
\end{section}

\newpage
\begin{section}*{Calendar Week: 04 \hfill \date{29 January, 2021}}

\begin{subsection}*{Completed Tasks}
    \begin{enumerate}
        \item 
            Finalized with the alpha version of health state estimation chapter.
        \item 
            Prepared minutes for weekly status meeting on 27 January 2021.
    \end{enumerate}
\end{subsection}

\begin{subsection}*{Tasks in-progress}
    \begin{enumerate}
        \item
           Preparing the review for health index chapter.
        
        \end{enumerate}
\end{subsection}

\end{section}
\newpage
\begin{section}*{Calendar Week: 03 \hfill \date{21 January, 2021}}

\begin{subsection}*{Completed Tasks}
    \begin{enumerate}
        \item 
            Updated the approaches such as DBN and logistic regression associated with health state estimation.
        \item
            Updated the evaluation metrics dealing with the classification models.
        \item
            Updated the introduction to health state chapter.
        
    \end{enumerate}
\end{subsection}

\begin{subsection}*{Tasks in-progress}
    \begin{enumerate}
        \item
           Finalizing the health state chapter for review.
        
        \end{enumerate}
\end{subsection}

\end{section}
\newpage
\begin{section}*{Calendar Week: 02 \hfill \date{15 January, 2021}}

\begin{subsection}*{Completed Tasks}
    \begin{enumerate}
        \item 
            Updated the formal definition of health state estimation as per review comments of Tanja.
        \item
            Updated the dataset part of survey report with regard to chapter health state estimation by removing the GHL dataset.
        \item
            Modified preliminary versions of state of the art approaches namely SVM.
        
    \end{enumerate}
\end{subsection}

\begin{subsection}*{Tasks in-progress}
    \begin{enumerate}
        \item
            To add on further details to approaches ANN and CART associated with supervised health state classification.
        \item
            To build the contents for evaluation for approaches involved.
        \end{enumerate}
\end{subsection}

\end{section}
\newpage
\begin{section}*{Calendar Week: 01 \hfill \date{10 January, 2021}}

\begin{subsection}*{Completed Tasks}
    \begin{enumerate}
        \item 
            Discussed with Saghar on the data sets to be added in the survey report for health state estimation.
        \item
            Prepared preliminary versions of state of the art approaches namely ANN, SVM and CART.
    \end{enumerate}
\end{subsection}
\begin{subsection}*{Tasks in-progress}
    \begin{enumerate}
        \item
            To dive deeper into the concepts of  different approaches (SVM , ANN and CART) associated with supervised health state classification.
        \item
            To build the contents for evaluation for approaches involved.
        \end{enumerate}
\end{subsection}
\begin{thebibliography}{9}
    \bibitem{survey81}
    {Hack-Eun Kim,  Andy C.C. Tan, Joseph Mathew  and  Byeong-Keun Choi},
    "Bearing fault prognosis based on health state probability estimation",
    \textit{Expert system with application 39, 2012},
    \href{https://www.sciencedirect.com/science/article/pii/S0957417411015491}{https://www.sciencedirect.com/science/article/pii/S0957417411015491}.
    \bibitem{survey82}
    {Hack-Eun Kim},
    "Machine prognosis based on health state probability estimation",
    \href{https://core.ac.uk/download/pdf/10903836.pdf}{https://core.ac.uk/download/pdf/10903836.pdf}.
    \end{thebibliography}
\end{section}
\newpage
\begin{section}*{Calendar Week: 52 - 53 \hfill \date{30 December, 2020}}

\begin{subsection}*{Completed Tasks}
    \begin{enumerate}
        \item 
            Discussed with Saghar on the corrections of formal definition as mentioned by Tanja and updated the same in the survey report.
        \item
            Had meeting with team member Saghar to discuss on the list of data sets for health state classification and found the following dataset-
            \begin{itemize}
                \item Condition monitoring of hydraulic systems \cite{survey71}  
                \item Bearing Dataset from NASA
                \item Bearing Fault dataset \cite{survey72}
            \end{itemize}
             
    \end{enumerate}
\end{subsection}
\begin{subsection}*{Tasks in-progress}
    \begin{enumerate}
        \item
            Getting more insights on different approaches (SVM , ANN and CART) associated with supervised health state classification.
        \item
            To build the contents for data sets and various approaches involved.
        \end{enumerate}
\end{subsection}
\begin{thebibliography}{9}
    \bibitem{survey71}
    {Nikolai  Helwig,  Eliseo  Pignanelli,  and  Andreas  Schutze},
    "Condition monitoring  of  a  complex  hydraulic  system  using  multivariate  statistics",
    \textit{2015 IEEE International Instrumentation and MeasurementTechnology Conference (I2MTC) Proceeding},
    \href{https://doi.org/10.1109/I2MTC.2015.7151267.1}{https://doi.org/10.1109/I2MTC.2015.7151267.1}. 
    
    \bibitem{survey72}
    {Eric Bechhoefer},
    "A quick introduction to bearing envelope analysis",
    \textit{ Green Power Monitoring Systems}
    \href{https://www.mfpt.org/wordpress1/wp-content/uploads/2017/11/mfpt-bearing-envelope-analysis-1.pdf}{https://www.mfpt.org/wordpress1/wp-content/uploads/2017/11/mfpt-bearing-envelope-analysis-1.pdf}.

    \end{thebibliography}
\end{section}


\newpage
\begin{section}*{Calendar Week: 51 \hfill \date{18 December, 2020}}

\begin{subsection}*{Completed Tasks}
    \begin{enumerate}
        \item 
            Discussed with team on the formal definition of time series as per \cite{survey73} and updated the same in the topic-study survey report.
        \item
            Had meeting with team member Saghar to discuss on pipeline structure health state classification section in topic survey.
        \item
            Discussed with Saghar on the plan for second milestone to prepare plan-chart.
        \item
            Discussed with Saghar and created formal definition of health state estimation.
    \end{enumerate}
\end{subsection}
\begin{subsection}*{Tasks in-progress}
    \begin{enumerate}
        \item
            Getting deep understanding of various health state estimation approaches to prepare the data-sets involved in the survey report. 
        \item
            Working with team member Saghar on tasks such as list of data-sets and state-of-the-art approaches associated with health state classification to be added for the topic-study survey report.
    \end{enumerate}
\end{subsection}
\begin{thebibliography}{9}
    \bibitem{survey73}
    {Tornede, Tanja and Tornede, Alexander and Wever, Marcel and Mohr, Felix and H{\"u}llermeier, Eyke},
    "AutoML for Predictive Maintenance: One Tool to RUL them all",
     \textit{IoTStream \@ ECMLPKDD 2020},
    2020. 
    
    \end{thebibliography}
\end{section}



\newpage

\begin{section}*{Calendar Week: 50 \hfill \date{11 December, 2020}}

\begin{subsection}*{Completed Tasks}
    \begin{enumerate}
        \item 
            Discussed with team on the contents of introduction of topic-study survey report.
        \item
            Had meeting with team member Saghar to discuss on sub-sections and its assignee for health state classification section in topic survey.
        \item
            Got an understanding of paper\cite{survey62}.The survey mentioned the following-
            \begin{itemize}
                \item Usage of Hidden Markov Models (HMM) in determining unobservant health states from the observable sensor data. 
                \item Lead of Hierarchical HMM (HHMM) over HMM in representing multiple health states along with their state transition properties.
                \item Detailed description of HMM and HMMM in the area of diagnosis and prognosis of machinery parts such as drill-bits.
            \end{itemize}
             
    \end{enumerate}
\end{subsection}
\begin{subsection}*{Tasks in-progress}
    \begin{enumerate}
        \item
            Getting more insights from the seed literature\cite{survey62} and \cite{survey63} for getting more insights into topic of health state classification.
        \item
            To state formal definition of time series.
        \item
            Working with team member Saghar on tasks such as deciding the formal definition, pipeline elements associated with health state classification to be added for the topic-study survey report.
    \end{enumerate}
\end{subsection}
\begin{thebibliography}{9}
    \bibitem{survey61}
    {O. Geramifard, J.-X. Xu, C. K. Pang, J.H. Zhou, X. Li},
    "Data-Driven Approaches in Health Condition Monitoring – A
    Comparative Study",
    \textit{2010 8th IEEE International Conference on
    Control and Automation},
    \href{https://ieeexplore.ieee.org/stamp/stamp.jsp?tp= \&\ arnumber=5524339}{https://ieeexplore.ieee.org/stamp/stamp.jsp?tp= \&\ arnumber=5524339}. 
    
    \bibitem{survey62}
    {Camci, Fatih, and Ratna Babu Chinnam},
    "Health-state estimation and prognostics in machining processes",
    \textit{ IEEE Transactions on automation science and engineering}
    7.3 (2010): 581-597,
    \href{https://ieeexplore.ieee.org/stamp/stamp.jsp?arnumber=5393023}{https://ieeexplore.ieee.org/stamp/stamp.jsp?arnumber=5393023}.

    \bibitem{survey63}
    {{Kim, Hack-Eun, et} al.},
    "Bearing fault prognosis based on health state probability estimation.",
    Expert Systems with Applications 39.5 (2012): 5200-5213. 13,
    no. 3,
    \href{https://www.sciencedirect.com/science/article/pii/S0957417411015491}{https://www.sciencedirect.com/science/article/pii/S0957417411015491}.

    \end{thebibliography}
\end{section}

\newpage

\begin{section}*{Calendar Week: 49 \hfill \date{4 December, 2020}}

\begin{subsection}*{Completed Tasks}
    \begin{enumerate}
        \item 
            Discussed with team on the final template of topic-study survey report.
        \item
            Acquired brief knowledge on basic steps in health state classification from dissertation \cite{survey51}. The survey mentioned the following -
            \begin{itemize}
                \item Prognostics and health management (PHM) being combination of diagnostics (fault prediction, isolation and identification), prognostics (dealing with estimation the time to failure) and decision module (predicting remaining useful time) associated with condition monitoring.  
                \item Fault diagnosis is linked to feature extraction as its gets its input from feature extraction module and it provides model describing current health state for prognosis (health index estimation) and decision modules.
                \item Diagnostic methods are mainly classified as model-based and data-driven. 
                \item Model driven methods are based on calculation of residue which is deviation of at least one determining property from its acceptable behavior. 
                \item Data driven fault prediction mainly involves use of pattern recognition that could be done through supervised or unsupervised learning approaches.  
            \end{itemize}
             
    \end{enumerate}
\end{subsection}
\begin{subsection}*{Tasks in-progress}
    \begin{enumerate}
        \item
            Getting more insights from the seed literature\cite{survey52} and \cite{survey53} provided as part of second milestone for deep diving into topic of health state classification.
        \item
            Decide upon subsections of health state classification report and its assignee.
    \end{enumerate}
\end{subsection}
\begin{thebibliography}{9}
    \bibitem{survey51}
    {J. K. {Kimotho}},
    "Development and performance evaluation of prognostic approaches for technical systems",
    \textit{ Ph.D. dissertation},
    2016,
    \href{ https://digital.ub.uni-paderborn.de/hsx/content/titleinfo/2219021}{ https://digital.ub.uni-paderborn.de/hsx/content/titleinfo/2219021}.
    
    \bibitem{survey52}
    {Camci, Fatih, and Ratna Babu Chinnam},
    "Health-state estimation and prognostics in machining processes",
    \textit{ IEEE Transactions on automation science and engineering}
    7.3 (2010): 581-597,
    \href{https://ieeexplore.ieee.org/stamp/stamp.jsp?arnumber=5393023}{https://ieeexplore.ieee.org/stamp/stamp.jsp?arnumber=5393023}.

    \bibitem{survey53}
    {{Kim, Hack-Eun, et} al.},
    "Bearing fault prognosis based on health state probability estimation.",
    Expert Systems with Applications 39.5 (2012): 5200-5213. 13,
    no. 3,
    \href{https://www.sciencedirect.com/science/article/pii/S0957417411015491}{https://www.sciencedirect.com/science/article/pii/S0957417411015491}.

    \end{thebibliography}
\end{section}

\newpage

\begin{section}*{Calendar Week: 48 \hfill \date{27 November, 2020}}

\begin{subsection}*{Completed Tasks}
    \begin{enumerate}
        \item
            Finished with understanding of Survey paper 2\cite{survey31}. The survey mentioned the following -
            \begin{itemize}
                \item The three predictive maintenance(PdM) prognosis methods namely model-based, knowledge-based and data-driven wherein data-driven techniques lead over other two maintenance strategies.  
                \item Comparison between machine learning and deep learning models. Data-driven machine learning models have sub-processes such as feature extraction which is avoided by deep learning models through addition of complex layers between the input raw data and required prediction result.
                \item Predictive maintenance being significant component of prognostic and health management of industrial equipment. 
                \item Operational assessment determining the success rate of predictive maintenance approaches.
                \item Potential improvements in the field of predictive maintenance including data validity, unbalanced nature of dataset, model migration and generalization ability, unsupervised learning and safety parameters.  
            \end{itemize}
             
    \end{enumerate}
\end{subsection}
\begin{subsection}*{Tasks in-progress}
    \begin{enumerate}
        \item
            Collecting more information on recent fault prediction techniques in the area of predictive maintenance. 
    \end{enumerate}
\end{subsection}
\begin{thebibliography}{9}
    \bibitem{survey31}
    {W. {Zhang} and D. {Yang} and H. {Wang}},
    "Data-Driven Methods for Predictive Maintenance of Industrial Equipment: A Survey",
    \textit{IEEE Systems Journal},
    vol. 13,
    no. 3,
    pp. 2213-2227,
    2019,
    \href{https://ieeexplore.ieee.org/document/8707108}{https://ieeexplore.ieee.org/document/8707108}.

    \end{thebibliography}
\end{section}

\newpage

\begin{section}*{Calendar Week: 47 \hfill \date{19 November, 2020}}

\begin{subsection}*{Completed Tasks}
    \begin{enumerate}
        \item
            Completed going through of Survey paper 1 \cite{survey11}. The survey described the following -
            \begin{itemize}
                \item Lead of predictive maintenance(PdM) over reactive and preventive maintenance strategies through utilisation of latest evolving technologies such as internet of things(for data acquisition),  big data(for data pre-processing), advanced deep learning(for fault diagnosis and prognosis) and deep reinforcement learning(for decision making).  
                \item Different types of architectures associated with predictive maintenance such as Open System Architecture for Condition based monitoring, cloud based PdM system and PdM 4.0.
                \item Objectives of PdM include getting rid of abrupt downtime and making system more reliable, reduce maintenance costs involved and fix contradicting multi-objective decision scenarios.
                \item Major categories of approaches involved in PdM - knowledge based, traditional machine learning(ML) based and deep learning(DL) based, of which DL based approaches apparently found to be efficient that the other two approaches.
                \item Challenges faced by PdM such as complexity, higher level of automation and constantly changing nature of current industrial systems.  
            \end{itemize}
             
    \end{enumerate}
\end{subsection}
\begin{subsection}*{Tasks in-progress}
    \begin{enumerate}
        \item
            Collecting information from second survey paper\cite{survey21} which mainly explains major applications of PdM in different industrial scenarios.
    \end{enumerate}
\end{subsection}
\begin{thebibliography}{9}
    \bibitem{survey11}
    {Y. Ran, X. Zhou, P. Lin, Y. Wen, and R. Deng},
    "A survey of predictive maintenance: Systems, purposes and approaches",
    \textit{arXiv preprint arXiv:1912.07383},
    2019,
    \href{https://arxiv.org/pdf/1912.07383.pdf}{https://arxiv.org/pdf/1912.07383.pdf}.
    \bibitem{survey21}
    {W. {Zhang} and D. {Yang} and H. {Wang}},
    "Data-Driven Methods for Predictive Maintenance of Industrial Equipment: A Survey",
    \textit{IEEE Systems Journal},
    vol. 13,
    no. 3,
    pp. 2213-2227,
    2019,
    \href{https://ieeexplore.ieee.org/document/8707108}{https://ieeexplore.ieee.org/document/8707108}.

    \end{thebibliography}
\end{section}

\newpage

\begin{section}*{Calendar Week: 46 \hfill \date{13 November, 2020}}

\begin{subsection}*{Completed Tasks}
    \begin{enumerate}
        \item
            Created minutes for meeting conducted on 11 November 2020. 
        \item
            Completed understanding of Survey paper 3 \cite{survey2}. The survey described following points -
            \begin{itemize}
                \item Increasing significance of predictive maintenance in the industrial domain among different types of maintenance strategies.  
                \item Dominating nature of artificial intelligence approaches (machine learning techniques) for maintenance over statistical and model-based approaches.
                \item Dependency of the performance of predictive maintenance over type of machine learning models used. 
                \item Various machine learning models involved, their merits and demerits and different types of data sets for predictive maintenance.  
            \end{itemize}
             
    \end{enumerate}
\end{subsection}
\begin{subsection}*{Tasks in-progress}
    \begin{enumerate}
        \item
            Reading through first survey paper\cite{survey1}.
            Going through introduction gave information on different architectures,objectives and methodologies involved in the area of predictive maintenance via machine learning techniques.
 
    \end{enumerate}
\end{subsection}
\begin{thebibliography}{9}
    \bibitem{survey1}
    {Y. Ran, X. Zhou, P. Lin, Y. Wen, and R. Deng},
    "A survey of predictive maintenance: Systems, purposes and approaches",
    \textit{arXiv preprint arXiv:1912.07383},
    2019,
    \href{https://arxiv.org/pdf/1912.07383.pdf}{https://arxiv.org/pdf/1912.07383.pdf}.
    \bibitem{survey2}
    {T. P. Carvalho and F. Soares and R. Vita and R. Francisco and Jo{\~a}o P. Basto and Symone G.S. Alcal{\'a}},
    “A systematic literature review of machine learning methods applied to predictive maintenance",
    \textit{Computers \&\ Industrial Engineering},
    vol. 137,
    p. 106024,
    2019.
\end{thebibliography}
\end{section}
\newpage

\begin{section}*{Calendar Week: 45 \hfill \date{06 November, 2020}}

\begin{subsection}*{Completed Tasks}
    \begin{enumerate}
        \item
            Connected with team for tasks such as template creation for weekly status report and JIRA setup and learned about initial set up process.
        \item
            Went through document shared by Tanja on Introduction to predictive maintenance wherein there is brief overview on various types of maintenance strategies involved, the reason behind lead of predictive maintenance over others and major steps, targets and data involved in predictive maintenance .

    \end{enumerate}
\end{subsection}
\begin{subsection}*{Challenges}
     \begin{enumerate}
         \item
            Creating weekly status report on Latex took more time than expected time.
        
\end{enumerate}
\end{subsection}
\begin{subsection}*{Tasks in-progress}
    \begin{enumerate}
        \item
            Collecting insights from third survey paper \cite{survey3} Initial reading gave the importance of predictive maintenance in the industrial field by means of systematic analysis of data from various sources through machine learning techniques.
 
    \end{enumerate}
\end{subsection}
\begin{thebibliography}{9}
    \bibitem{survey3}
    {T. P. Carvalho and F. Soares and R. Vita and R. Francisco and Jo{\~a}o P. Basto and Symone G.S. Alcal{\'a}},
    “A systematic literature review of machine learning methods applied to predictive maintenance",
    \textit{Computers \&\ Industrial Engineering},
    vol. 137,
    p. 106024,
    2019.
\end{thebibliography}
\end{section}

\end{document}