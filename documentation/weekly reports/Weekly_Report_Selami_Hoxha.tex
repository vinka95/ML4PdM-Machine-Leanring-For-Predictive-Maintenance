\documentclass[11pt,a4paper]{article}
\usepackage{graphicx}
\usepackage[utf8]{inputenc}
\usepackage{fancyhdr}
\usepackage{hyperref}
\usepackage[sorting=none, backend=biber]{biblatex}
\usepackage{filecontents}
\begin{filecontents}[overwrite,nosearch]{references.bib}
@article{DBLP:journals/corr/abs-1912-07383,
    author    = {Yongyi Ran and
                 Xin Zhou and
                 Pengfeng Lin and
                 Yonggang Wen and
                 Ruilong Deng},
    title     = {A Survey of Predictive Maintenance: Systems, Purposes and Approaches},
    journal   = {CoRR},
    volume    = {abs/1912.07383},
    year      = {2019},
    url       = {http://arxiv.org/abs/1912.07383},
    archivePrefix = {arXiv},
    eprint    = {1912.07383},
    timestamp = {Wed, 01 Jul 2020 09:22:04 +0200},
    biburl    = {https://dblp.org/rec/journals/corr/abs-1912-07383.bib},
    bibsource = {dblp computer science bibliography, https://dblp.org}
}
%
@article{DBLP:journals/sj/ZhangYW19,
    author    = {Weiting Zhang and
                Dong Yang and
                Hongchao Wang},
    title     = {Data-Driven Methods for Predictive Maintenance of Industrial Equipment:
                {A} Survey},
    journal   = {{IEEE} Syst. J.},
    volume    = {13},
    number    = {3},
    pages     = {2213--2227},
    year      = {2019},
    url       = {https://doi.org/10.1109/JSYST.2019.2905565},
    doi       = {10.1109/JSYST.2019.2905565},
    timestamp = {Fri, 11 Sep 2020 15:01:32 +0200},
    biburl    = {https://dblp.org/rec/journals/sj/ZhangYW19.bib},
    bibsource = {dblp computer science bibliography, https://dblp.org}
}
%
@article{DBLP:journals/candie/CarvalhoSVFBA19,
    author    = {Thyago Peres Carvalho and
                Fabr{\'{\i}}zzio Alphonsus A. M. N. Soares and
                Roberto Vita and
                Roberto da Piedade Francisco and
                Jo{\~{a}}o P. Basto and
                Symone G. S. Alcal{\'{a}}},
    title     = {A systematic literature review of machine learning methods applied
                to predictive maintenance},
    journal   = {Comput. Ind. Eng.},
    volume    = {137},
    year      = {2019},
    url       = {https://doi.org/10.1016/j.cie.2019.106024},
    doi       = {10.1016/j.cie.2019.106024},
    timestamp = {Mon, 26 Oct 2020 08:24:22 +0100},
    biburl    = {https://dblp.org/rec/journals/candie/CarvalhoSVFBA19.bib},
    bibsource = {dblp computer science bibliography, https://dblp.org}
}
%
@article{survey5,
author = {Riaz, Saleem and Elahi, Hassan and Javaid, Kashif and Shahzad, Tufail},
year = {2017},
month = {03},
pages = {103-110},
title = {Vibration Feature Extraction and Analysis for Fault Diagnosis of Rotating Machinery-A Literature Survey},
volume = {5},
journal = {Asia Pacific Journal of Multidisciplinary Research}
}
%
@ARTICLE{8097036,  
author={M. {Islam} and G. {Lee} and S. N. {Hettiwatte} and K. {Williams}},  
journal={IEEE Transactions on Power Delivery},   
title={Calculating a Health Index for Power Transformers Using a Subsystem-Based GRNN Approach},   
year={2018},  
volume={33},  
number={4},  
pages={1903-1912},  
doi={10.1109/TPWRD.2017.2770166}}
%
@ARTICLE{7377088,  
author={F. {Yang} and M. S. {Habibullah} and T. {Zhang} and Z. {Xu} and P. {Lim} and S. {Nadarajan}},  
journal={IEEE Transactions on Industrial Electronics},   
title={Health Index-Based Prognostics for Remaining Useful Life Predictions in Electrical Machines},   
year={2016},  
volume={63},  
number={4},  
pages={2633-2644},  
doi={10.1109/TIE.2016.2515054}}
%
@misc{malhotra2016multisensor,
      title={Multi-Sensor Prognostics using an Unsupervised Health Index based on LSTM Encoder-Decoder}, 
      author={Pankaj Malhotra and Vishnu TV and Anusha Ramakrishnan and Gaurangi Anand and Lovekesh Vig and Puneet Agarwal and Gautam Shroff},
      year={2016},
      eprint={1608.06154},
      archivePrefix={arXiv},
      primaryClass={cs.LG}
}
@misc{kimotoprog,
      title={Multi-Sensor Prognostics using an Unsupervised Health Index based on LSTM Encoder-Decoder}, 
      author={Pankaj Malhotra and Vishnu TV and Anusha Ramakrishnan and Gaurangi Anand and Lovekesh Vig and Puneet Agarwal and Gautam Shroff},
      year={2016},
      eprint={1608.06154},
      archivePrefix={arXiv},
      primaryClass={cs.LG}
}
@ARTICLE{jkkimothoprog,  
author={J.K. {Kimotho}},
journal = {Ph.D. dissertation},
title={Development and performance evaluation of prognostic approaches for technical systems},   
year={2016}
}
@ARTICLE{6518966,
  author={A. D. {Ashkezari} and H. {Ma} and T. K. {Saha} and C. {Ekanayake}},
  journal={IEEE Transactions on Dielectrics and Electrical Insulation}, 
  title={Application of fuzzy support vector machine for determining the health index of the insulation system of in-service power transformers}, 
  year={2013},
  volume={20},
  number={3},
  pages={965-973},
  doi={10.1109/TDEI.2013.6518966}}

  @inproceedings{DBLP:conf/indin/AremuOHM19,
  author    = {Oluseun Omotola Aremu and
               Darren O. O'Reilly and
               David Hyland{-}Wood and
               Peter Ross McAree},
  title     = {Kullback-Leibler Divergence Constructed Health Indicator for Data-Driven
               Predictive Maintenance of Multi-Sensor Systems},
  booktitle = {17th {IEEE} International Conference on Industrial Informatics, {INDIN}
               2019, Helsinki, Finland, July 22-25, 2019},
  pages     = {1315--1320},
  publisher = {{IEEE}},
  year      = {2019},
  url       = {https://doi.org/10.1109/INDIN41052.2019.8972069},
  doi       = {10.1109/INDIN41052.2019.8972069},
  timestamp = {Fri, 27 Mar 2020 08:58:39 +0100},
  biburl    = {https://dblp.org/rec/conf/indin/AremuOHM19.bib},
  bibsource = {dblp computer science bibliography, https://dblp.org}
}

\end{filecontents}
\addbibresource{references.bib}

\title{Project Group:\\
\textbf{Machine Learning for Predictive Maintenance}\\
(Weekly Status Report)}

\author{Supervisors:\\
\textbf{Prof. Dr. Eyke H{\"u}llermeier (eyke@upb.de)}\\
\textbf{Tanja Tornede (tanja.tornede@upb.de)}\\ 
\vspace{10mm}
Name: Selami Hoxha \\
Matriculation Number: 6853870\\
E-mail: shoxha@mail.uni-paderborn.de
}
\date{}

% Definition of \maketitle
\makeatletter
\def \@maketitle{
\begin{center}
\includegraphics[width=10cm, height=3cm]{logo.png}\\
{\vspace{12mm} \@title}\\[4ex]
{\@author}\\[4ex]
\end{center}}
\makeatother

\pagestyle{fancy}
\setlength{\headheight}{14pt}
\rhead{PG ML4PdM}
\lhead{Selami Hoxha}

\begin{document}

\maketitle
\thispagestyle{empty}

\clearpage
\pagenumbering{arabic}

%If u want to add more pages? --> copy the below code and paste it here.
\newpage
\begin{section}*{Calendar Week: 10 \hfill \date{11 March, 2021}}
 \begin{refsection}

       \begin{subsection}*{Completed Tasks}
             \begin{enumerate}
                   \item
                         Created the first version of the class diagram for HIE
                   \item
                         Created the first version of the sequence diagrams for the three approaches.
                   \item
                         Together with the team discussed and made changes to the general class diagram.
                   \item
                         Wrote a short description of the general class diagram on the system design document.
             \end{enumerate}
       \end{subsection}

       \begin{subsection}*{Challenges}
             \begin{enumerate}
                   \item
                         When creating the class diagram it was hard to forsee which parameters will each class have.
             \end{enumerate}
       \end{subsection}

       \begin{subsection}*{Tasks in-progress}
             \begin{enumerate}
                   \item
                         Finalize the class diagram.
                   \item
                         Finalize the sequence diagram.
                   \item
                         Write a description on the system design document for the class diagram and sequence diagrams.
                   \item
                         Imporve the description of the general class diagram in the system design document.

             \end{enumerate}
       \end{subsection}
       %   \printbibliography
 \end{refsection}
\end{section}
% Page End here.

% If u want to add more pages? --> copy the below code and paste it here.
\newpage
\begin{section}*{Calendar Week: 9  \hfill \date{05 March, 2021}}
 \begin{refsection}

       \begin{subsection}*{Completed Tasks}
             \begin{enumerate}
                   \item Wrote the example part in the data format documentation
                   \item Together with the team worked on the class diagram.
             \end{enumerate}
       \end{subsection}

       \begin{subsection}*{Tasks in-progress}
             \begin{enumerate}
                   \item Finalize the data format chapter.
                   \item Add the class diagrams for the HI chapter.
                   \item Create the sequence diagrams for the HI chapter.
             \end{enumerate}
       \end{subsection}

       %  \printbibliography
 \end{refsection}
\end{section}
% Page End here.

% If u want to add more pages? --> copy the below code and paste it here.
\newpage
\begin{section}*{Calendar Week: 8  \hfill \date{26 February, 2021}}
 \begin{refsection}

       \begin{subsection}*{Completed Tasks}
             \begin{enumerate}
                   \item Worked together with the team to come up with the initial data format for the ML4PdM framework.
                   \item Wrote the example part in the initial data format documentation.
                   \item Together with the team redefined the data format based on the feedback we got on the wednesday meeting of 24 February.
             \end{enumerate}
       \end{subsection}

       \begin{subsection}*{Tasks in-progress}
             \begin{enumerate}
                   \item Write the documentation for the defined data format.
                   \item Help the team in the framework design.
             \end{enumerate}
       \end{subsection}

       %  \printbibliography
 \end{refsection}
\end{section}
% Page End here.

% If u want to add more pages? --> copy the below code and paste it here.
\newpage
\begin{section}*{Calendar Week: 7  \hfill \date{19 February, 2021}}
 \begin{refsection}

       \begin{subsection}*{Completed Tasks}
             \begin{enumerate}
                   \item Reviewed survey chapters:
                         \begin{enumerate}
                               \item Introduction
                               \item Formal definition of time series
                               \item RUL
                               \item Conlusion
                               \item References
                         \end{enumerate}
                   \item
                         Made the final changes to the sruvey based on the second
                         review.
                   \item Checked my work for spelling and punctuation mistakes.
                   \item Fixed the dataset references to show in bibliography
                         as a seperate section of bibliography named Datasets (p. 110).
             \end{enumerate}
       \end{subsection}

       \begin{subsection}*{Tasks in-progress}
             \begin{enumerate}
                   \item Think and design a format for the data representation.
                   \item Think and help the team in design of an architecture for the ML4PdM framework.
             \end{enumerate}
       \end{subsection}

       %  \printbibliography
 \end{refsection}
\end{section}
% Page End here.

% If u want to add more pages? --> copy the below code and paste it here.
\newpage
\begin{section}*{Calendar Week: 6  \hfill \date{12 February, 2021}}
 \begin{refsection}

       \begin{subsection}*{Completed Tasks}
             \begin{enumerate}
                   \item
                         Completed changes based on feedback for approaches and publicly avaliable datasets
                         of the HI chapter. The changes were mostly about fixing
                         errors, but also included modifying or adding some
                         content.
                   \item
                         Added a very brief introduction of Principal Component
                         Analysis.
             \end{enumerate}
       \end{subsection}

       \begin{subsection}*{Tasks in-progress}
             \begin{enumerate}
                   \item Review survey chapters:
                         \begin{enumerate}
                               \item Introduction
                               \item Formal definition of time series
                               \item RUL
                               \item Conlusion
                               \item References
                         \end{enumerate}
                   \item
                         Implement the changes in the evaluation section.
                   \item
                         Implement the changes based on the second review to
                         get the final version of the survey.
             \end{enumerate}
       \end{subsection}

       %  \printbibliography
 \end{refsection}
\end{section}
% Page End here.


%If u want to add more pages? --> copy the below code and paste it here.
\newpage
\begin{section}*{Calendar Week: 5  \hfill \date{5 February, 2021}}
 \begin{refsection}

       \begin{subsection}*{Completed Tasks}
             \begin{enumerate}
                   \item
                         Completed the revision for the alpha version of ML4PdM survey
                         for chapters:
                         \begin{enumerate}
                               \item Introduction
                               \item Formal definition of time series
                               \item Health state classification
                               \item Conlusion
                               \item References
                         \end{enumerate}
                   \item
                         Made changes based on feeback I received for the Publicly Available Dataset chapter.
             \end{enumerate}
       \end{subsection}

       \begin{subsection}*{Tasks in-progress}
             \begin{enumerate}
                   \item
                         Make further changes based on feedback for the other parts I have written. That
                         includes three approaches and evaluation for HI chapter; also the paragraph for
                         the history of predictive maintenance in the Introduction chapter.
             \end{enumerate}
       \end{subsection}

       %  \printbibliography
 \end{refsection}
\end{section}
% Page End here.


%If u want to add more pages? --> copy the below code and paste it here.
\newpage
\begin{section}*{Calendar Week: 4  \hfill \date{29 January, 2021}}
 \begin{refsection}

       \begin{subsection}*{Completed Tasks}
             \begin{enumerate}
                   \item
                         Completed writing the Kullback-Leibler approach for multi-sensor prognostics \cite{DBLP:conf/indin/AremuOHM19}.
                   \item
                         Completed the evaluation chapter and the publicly available data chapter.
             \end{enumerate}
       \end{subsection}

       \begin{subsection}*{Tasks in-progress}
             \begin{enumerate}
                   \item
                         Evaluation of the alpha version of the survey. The chapters
                         that I will cover are:
                         \begin{enumerate}
                               \item Introduction
                               \item Formal definition of Time series
                               \item Health state chapter
                               \item Conlusion chapter
                         \end{enumerate}
             \end{enumerate}
       \end{subsection}

       \printbibliography
 \end{refsection}
\end{section}
% Page End here.

%If u want to add more pages? --> copy the below code and paste it here.
\newpage
\begin{section}*{Calendar Week: 3  \hfill \date{22 January, 2021}}
 \begin{refsection}

       \begin{subsection}*{Completed Tasks}
             \begin{enumerate}
                   \item
                         Replaced the third approach with an appropriate one.
             \end{enumerate}
       \end{subsection}

       \begin{subsection}*{Tasks in-progress}
             \begin{enumerate}
                   \item
                         Add dataset for GRNN and ANN approaches.
                   \item
                         Fill the evaluation module with the missing metrics.
                   \item
                         Revise all the content written for spelling and notation mistakes.
                   \item
                         Check the references for correctness.
             \end{enumerate}
       \end{subsection}

       %   \printbibliography
 \end{refsection}
\end{section}
% Page End here.

%If u want to add more pages? --> copy the below code and paste it here.
\newpage
\begin{section}*{Calendar Week: 2 \hfill \date{15 January, 2021}}
 \begin{refsection}

       \begin{subsection}*{Completed Tasks}
             \begin{enumerate}
                   \item
                         Added new content in Evaluation chapter.
                   \item
                         Added the first datasets in the the public available datasets chapter.
                   \item
                         Modified some notation in the LSTM-ED approach to correspond with the latest
                         introduced notation for a single time step in a time series.
             \end{enumerate}
       \end{subsection}

       \begin{subsection}*{Challenges}
             \begin{enumerate}
                   \item
                         It is challenging to adapt the notation for subsequence of a time series in
                         the way the approach is presented. It is something that might need a bit more
                         thinking in the review phase.
                   \item
                         Have decided that the fuzzy support vector machine approach will not work. The
                         HI from expert can not be obtained. Hence a replacement should be found.
             \end{enumerate}
       \end{subsection}

       \begin{subsection}*{Tasks in-progress}
             \begin{enumerate}
                   \item
                         Find more data sets that can be used by the approaches.
                   \item
                         Complete the evaluation setup.
                   \item
                         Replace fuzzy support vector machine approach.
             \end{enumerate}
       \end{subsection}

       %   \printbibliography
 \end{refsection}
\end{section}
% Page End here.

%If u want to add more pages? --> copy the below code and paste it here.
\newpage
\begin{section}*{Calendar Week: 1 \hfill \date{08 January, 2021}}
 \begin{refsection}

       \begin{subsection}*{Completed Tasks}
             \begin{enumerate}
                   \item
                         Completed the Fuzzy support vector machines approach.
                   \item
                         Completed the hierarchical gated unit network approach.
             \end{enumerate}
       \end{subsection}

       \begin{subsection}*{Challenges}
             \begin{enumerate}
                   \item
                         During writing some questions about notation came up, which should be clarified in the upcoming meetings.
             \end{enumerate}
       \end{subsection}

       \begin{subsection}*{Tasks in-progress}
             \begin{enumerate}
                   \item
                         Extract the description for publicly available data.
                   \item
                         Write the Evaluation Setup section.
             \end{enumerate}
       \end{subsection}

       %   \printbibliography
 \end{refsection}
\end{section}
% Page End here.


%If u want to add more pages? --> copy the below code and paste it here.
\newpage
\begin{section}*{Calendar Week: 52-53 \hfill \date{30 December, 2020}}
 \begin{refsection}

       \begin{subsection}*{Completed Tasks}
             \begin{enumerate}
                   \item
                         Helped Gourav Prakash complete the formal definition of health index.
                   \item
                         Completed the writing of the TLSM encoder decoder approach.
             \end{enumerate}
       \end{subsection}

       \begin{subsection}*{Challenges}
             \begin{enumerate}
                   \item
                         The fuzzy support vector machine approach for determining health index uses discrete values from 1-5 instead
                         of real values from 0 up to 1 as we defined it in the formal definition. The challenge is to adapt the formal
                         definition so it captures this case as presented in \cite{6518966}
             \end{enumerate}
       \end{subsection}

       \begin{subsection}*{Tasks in-progress}
             \begin{enumerate}
                   \item
                         Write about health index Based prognostics using Ensemble of machine learning algorithms.
                   \item
                         Write about determining health index of a system using fuzzy support vector machines.
             \end{enumerate}
       \end{subsection}

       \printbibliography
 \end{refsection}
\end{section}
% Page End here.

%If u want to add more pages? --> copy the below code and paste it here.
\newpage
\begin{section}*{Calendar Week: 51 \hfill \date{18 December, 2020}}
 \begin{refsection}

       \begin{subsection}*{Completed Tasks}
             \begin{enumerate}
                   \item
                         Have written history of Predictive maintenance. Some improvements are still necessary.
                   \item
                         Finalized the plan for the upcoming weeks.
             \end{enumerate}
       \end{subsection}

       %    \begin{subsection}*{Challenges}
       %          \begin{enumerate}
       %                \item
       %                     A small challenge has been finding papers that focus soley on health index calculations. 
       %                     A lot of papers have health index as an itermedeiate result needed for RUL calculation.
       %          \end{enumerate}
       %    \end{subsection}

       \begin{subsection}*{Tasks in-progress}
             \begin{enumerate}
                   \item
                         Complete the health index formal definition chapter.
                   \item
                         Write the first Approach for calculating HI.
             \end{enumerate}
       \end{subsection}

       %       \printbibliography
 \end{refsection}
\end{section}
% Page End here.

%If u want to add more pages? --> copy the below code and paste it here.
\newpage
\begin{section}*{Calendar Week: 50 \hfill \date{11 December, 2020}}
 \begin{refsection}

       \begin{subsection}*{Completed Tasks}
             \begin{enumerate}
                   \item
                         Done with reading of the seed papers \cite{8097036}, \cite{7377088}, and \cite{malhotra2016multisensor} and
                         geatherd material that explain different machine learning methods for estimating health index, which will be
                         studied in detail in the upcoming weeks.
                   \item
                         Completed the partition of the work on the heath index section with Gourav Prakash.
             \end{enumerate}
       \end{subsection}

       \begin{subsection}*{Challenges}
             \begin{enumerate}
                   \item
                         A small challenge has been finding papers that focus soley on health index calculations.
                         A lot of papers have health index as an itermedeiate result needed for RUL calculation.
             \end{enumerate}
       \end{subsection}

       \begin{subsection}*{Tasks in-progress}
             \begin{enumerate}
                   \item
                         Write down the plan for the upcoming seven weeks.
                   \item
                         Start with the first two methods for calculating health index, Subsystem-Based GRNN
                         \cite{8097036} and Neural Networks \cite{7377088}.
                   \item Write the history of predictive maintenance part for the Introduction chapter.
             \end{enumerate}
       \end{subsection}

       \printbibliography
 \end{refsection}
\end{section}
% Page End here.

%If u want to add more pages? --> copy the below code and paste it here.
\newpage
\begin{section}*{Calendar Week: 49 \hfill \date{04 December, 2020}}
 \begin{refsection}

       \begin{subsection}*{Completed Tasks}
             \begin{enumerate}
                   \item Participated in the thursday meeting where we decided the primary structure of the survey document.
                   \item Had an initial meeting with Gourav Prakash where we discussed our intial thougts
                         for a plan for milestone 2.
             \end{enumerate}
       \end{subsection}

       \begin{subsection}*{Challenges}
             \begin{enumerate}
                   \item Deciding on the pipleine elements that will be included.
             \end{enumerate}
       \end{subsection}

       \begin{subsection}*{Tasks in-progress}
             \begin{enumerate}
                   \item Seperation of the work for the subtopic.
                   \item Give a short description for each section of Health Index estimation subtopic.
                   \item Read the seed papers \cite{8097036}, \cite{7377088}, and \cite{malhotra2016multisensor} and start gathering literature
                         for other methods.
             \end{enumerate}
       \end{subsection}

       \printbibliography
 \end{refsection}
\end{section}
% Page End here.

%If u want to add more pages? --> copy the below code and paste it here.
\newpage
\begin{section}*{Calendar Week: 48 \hfill \date{27 November, 2020}}
 \begin{refsection}

       \begin{subsection}*{Completed Tasks}
             \begin{enumerate}
                   \item
                         In this week I looked up some extra information on how feature extraction can be achieved. More precisely I
                         looked at \emph{Vibration Feature Extraction Techniques for Fault Diagnosis of Rotating Machinery -
                               A Literature Survey}. The focus of this survey were techniques for feature extraction from vibration data
                         in rotating machinery. Techniques for extraction in time domain, frequency domain and time-frequency
                         domain were presented. Also, their performance was evaluated. \cite{survey5}
             \end{enumerate}
       \end{subsection}

       \begin{subsection}*{Challenges}
             \begin{enumerate}
                   \item
                         If the subtopic I will be working on is feature extraction, then a few of the techniques presented
                         in \cite{survey5} will need a bit more attention and understanding.
             \end{enumerate}
       \end{subsection}

       \printbibliography
 \end{refsection}
\end{section}
% Page End here.

%If u want to add more pages? --> copy the below code and paste it here.
\newpage
\begin{section}*{Calendar Week: 47 \hfill \date{20 November, 2020}}
 \begin{refsection}

       \begin{subsection}*{Completed Tasks}
             \begin{enumerate}
                   \item
                         Read \emph{Development and Performance Evaluation of Prognostic Approaches for Technical Systems} by James Kuria
                         Kimotho up to chapter 4. The work covers in a detailed manner the prognostic approaches. Initially an overview
                         of diagnostic and prognostic approaches is 	given (chapter 1 and 2). Then, methodologies for preprocessing
                         of the data, feature extraction, and feature selection are presented (chapter 3). Lastly, methods for identifying
                         the health states, estimating health index and calculating RUL are given (chapter 4). Also, applications
                         of these methodologies and their results are presented. \cite{jkkimothoprog}
             \end{enumerate}
       \end{subsection}

       %\begin{subsection}*{Challenges}
       %    \begin{enumerate}
       %        \item
       %           
       %        \item
       %            Narrow down the choice of the subtopic.
       %    \end{enumerate}
       %\end{subsection}

       \begin{subsection}*{Tasks in-progress}
             \begin{enumerate}
                   \item
                         Research focused on feature extraction and health detection/health classification as a preparation for
                         the subtopic selection.
             \end{enumerate}
       \end{subsection}

       \printbibliography
 \end{refsection}
\end{section}
% Page End here.

%If u want to add more pages? --> copy the below code and paste it here.
\newpage
\begin{section}*{Calendar Week: 46 \hfill \date{13 November, 2020}}
 \begin{refsection}

       \begin{subsection}*{Completed Tasks}
             \begin{enumerate}
                   \item
                         Read the survey \emph{A Survey of Predictive Maintenance: Systems, Purposes and Approaches}.
                         This survey is focused on system architectures, purposes and approaches used for developing
                         a PdM system. Firstly, three types of architectures are described: Open System Architecture
                         for Condition Based Monitoring (OSA-CBM), Cloud enhanced PdM systems and PdM 4.0. Then,
                         cost minimization and reliability/availability maximization and are given as the purpose
                         of these systems. Lastly, PdM approaches are categorized in three categories: knowledge
                         based, machine learning (ML) based and deep learning (DL) based. Then an overview of
                         these methods is given, with more focus on the DL methods.
                         \cite{DBLP:journals/corr/abs-1912-07383}
                   \item
                         The survey \emph{A systematic literature review of machine learning methods applied to predictive
                               maintenance} gives an overview of the latest research being done in PdM. Also, there is given a
                         summary of the journals and conferences where these papers are published. Also, a summary of the
                         most recent ML methods that are used to achieve PdM is given. Furthermore, public data sets
                         for use in predictive maintenance are presented.   \cite{DBLP:journals/candie/CarvalhoSVFBA19}
             \end{enumerate}
       \end{subsection}

       \begin{subsection}*{Challenges}
             \begin{enumerate}
                   \item
                         Understanding some of the more complicated DL methods.
                   \item
                         Narrow down the choice of the subtopic.
             \end{enumerate}
       \end{subsection}

       \begin{subsection}*{Tasks in-progress}
             \begin{enumerate}
                   \item
                         Reading more about some of the mentioned  DL methods to get a basic understanding.
                   \item
                         Explore more on how feature extraction is realized.
             \end{enumerate}
       \end{subsection}

       \printbibliography
 \end{refsection}
\end{section}
% Page End here.

% New page start here.
\newpage
\begin{section}*{Calendar Week: 45 \hfill \date{06 November, 2020}}
 \begin{refsection}

       \begin{subsection}*{Completed Tasks}
             \begin{enumerate}
                   \item
                         Followed and supported the team, on a lengthy meeting, with the setting up of the working
                         tools we are going to use.
                   \item
                         Read the survey titled \emph{Data-Driven Methods for Predictive Maintenance of Industrial
                               Equipment: A survey}. Firstly, the survey presented an overview on maintenance methods. Then
                         the Data-Driven methods are presented as the most important methods for achieving effective
                         maintenance. The increase of importance of these methods comes as a result of many big
                         industrial projects that are happening around the world, ex. Industry 4.0 in Germany.
                         I also learned the structure of a data-driven PdM system and its challenges for
                         implementation. Lastly the paper presented Machine Learning and Deep Learning
                         methods that are used in various research papers for implementing such a system.
                         An analysis of these methods is given in terms of accuracy and the challenges faced.
                         \cite{DBLP:journals/sj/ZhangYW19}
             \end{enumerate}
       \end{subsection}

       \begin{subsection}*{Challenges}
             \begin{enumerate}
                   \item
                         Get used to the workflow of the project and improve the knowledge on the tools we are going
                         to use.
                   \item
                         Help the team further improve our work environment.
             \end{enumerate}
       \end{subsection}

       \begin{subsection}*{Tasks in-progress}
             \begin{enumerate}
                   \item
                         Read the other surveys.
                   \item
                         Research on how are some of the PdM problems are handled in practice.
             \end{enumerate}
       \end{subsection}

       \printbibliography
 \end{refsection}
\end{section}
% Page end here.

\end{document}