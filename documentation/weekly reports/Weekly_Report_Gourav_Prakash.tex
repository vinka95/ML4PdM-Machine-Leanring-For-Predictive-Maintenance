\documentclass[11pt,a4paper]{article}
\usepackage{graphicx}
\usepackage[utf8]{inputenc}
\usepackage{fancyhdr}
\usepackage{hyperref}

\title{Project Group:\\
\textbf{Machine Learning for Predictive Maintenance}\\
(Weekly Status Report)}

\author{Supervisors:\\
\textbf{Prof. Dr. Eyke H{\"u}llermeier (eyke@upb.de)}\\
\textbf{Tanja Tornede (tanja.tornede@upb.de)}\\ 
\vspace{10mm}
Name: Gourav Prakash \\
Matriculation Number: 6865914\\
E-mail: rextron@mail.uni-paderborn.de
}
\date{}

% Definition of \maketitle
\makeatletter
\def \@maketitle{
\begin{center}
\includegraphics[width=10cm, height=3cm]{logo.png}\\
{\vspace{12mm} \@title}\\[4ex]
{\@author}\\[4ex]
\end{center}}
\makeatother

\pagestyle{fancy}
\setlength{\headheight}{14pt}
\rhead{PG ML4PdM}
\lhead{Gourav Prakash}

\begin{document}

\maketitle
\thispagestyle{empty}

\clearpage
\pagenumbering{arabic}


% %If u want to add more pages? --> copy the below code and paste it here.
% \newpage
% \begin{section}*{Calendar Week: 46 \hfill \date{13 November, 2020}}

% \begin{subsection}*{Completed Tasks}
%     \begin{enumerate}
%         \item
%             Your contents goes here.
%         \item
%             Your contents goes here.
%         \item
%             Your contents goes here.
%     \end{enumerate}
% \end{subsection}

% \begin{subsection}*{Challenges}
%     \begin{enumerate}
%         \item
%             Your contents goes here.
%         \item
%             Your contents goes here.
%     \end{enumerate}
% \end{subsection}

% \begin{subsection}*{Tasks in-progress}
%     \begin{enumerate}
%         \item
%             Your contents goes here. example references \cite{survey2}.
%         \item
%             Your contents goes here.
%         \item
%             Your contents goes here.
%         \item
%             Your contents goes here.
%     \end{enumerate}
% \end{subsection}

% \begin{thebibliography}{9}
%     \bibitem{survey2}
%     {W. {Zhang} and D. {Yang} and H. {Wang}},
%     "Data-Driven Methods for Predictive Maintenance of Industrial Equipment: A Survey",
%     \textit{IEEE Systems Journal},
%     vol. 13,
%     no. 3,
%     pp. 2213-2227,
%     2019,
%     \href{https://ieeexplore.ieee.org/document/8707108}{https://ieeexplore.ieee.org/document/8707108}.
% \end{thebibliography}
% \end{section}
% % Page End here.

\begin{section}*{Calendar Week: 07 \hfill \date{19 February, 2021}}

    \begin{subsection}*{Completed Tasks}
        \begin{enumerate}
            \item
                Submission of the final Study Report.
            \item 
                Took minutes of meeting
        \end{enumerate}
    \end{subsection}
    
    
    \begin{subsection}*{Tasks in-progress}
        \begin{enumerate}
            \item Working on the milestone 3.
            \item Working with team on first version of general sequence diagram.
            \item Working with team on the first version of interface, class, and method definition.
        \end{enumerate}
    \end{subsection}
\end{section}
% Page End here.

\newpage
\begin{section}*{Calendar Week: 06 \hfill \date{12 February, 2021}}

    \begin{subsection}*{Completed Tasks}
        \begin{enumerate}
            \item
                Made changes based on the 1st Review, Approaches proper name, increase the sharpness of the image used, line alignment, removing extra space which make the text weird.
            \item 
                Understanding and implementing chapter tags.
        \end{enumerate}
    \end{subsection}
    
    \begin{subsection}*{Tasks in-progress}
        \begin{enumerate}
            \item Give Review on Chapter HSC.
            \begin{enumerate}
                \item Introduction, motivation, contents like approaches, real data, evaluation, implementation.
            \end{enumerate}
        \end{enumerate}
    \end{subsection}
\end{section}
% Page End here.

\newpage
\begin{section}*{Calendar Week: 05 \hfill \date{05 February, 2021}}

    \begin{subsection}*{Completed Tasks}
        \begin{enumerate}
            \item
                Minor changes HIE chapter.
            \item 
                Restructuring the HIE chapter.
            \item 
                Changing the resolution of the images.
        \end{enumerate}
    \end{subsection}
    
    \begin{subsection}*{Tasks in-progress}
        \begin{enumerate}
            \item Making necessary chances based on feedbacks (Christopher and Anurose).
        \end{enumerate}
    \end{subsection}
\end{section}
% Page End here.

% Page start's here
\newpage
\begin{section}*{Calendar Week: 04 \hfill \date{29 January, 2021}}

    \begin{subsection}*{Completed Tasks}
        \begin{enumerate}
            \item
                Writing a feedback for common section like Introduction, document structure, abstract, motivation and formal definition for survey paper PdM4ML.
        \end{enumerate}
    \end{subsection}
    
    \begin{subsection}*{Tasks in-progress}
        \begin{enumerate}
            \item providing a feedback for RUL chapter.
        \end{enumerate}
    \end{subsection}
\end{section}
% Page End here.

% Page start's here
\newpage
\begin{section}*{Calendar Week: 03 \hfill \date{22 January, 2021}}

    \begin{subsection}*{Completed Tasks}
        \begin{enumerate}
            \item
                Making basic correction and adjustment for HI chapter.
            \item
                Select a good resolution images for subsection topic in HI.
            \item
                Analysing the use of proper references.
        \end{enumerate}
    \end{subsection}
    
    \begin{subsection}*{Tasks in-progress}
        \begin{enumerate}
            \item
                Defining the third approach.
            \item
                Working on conclusion for whole survey papers.
            \item
                Making basic dhanges if necessary.
        \end{enumerate}
    \end{subsection}
\end{section}
% Page End here.

% Page start's here
\newpage
\begin{section}*{Calendar Week: 02 \hfill \date{15 January, 2021}}

\begin{subsection}*{Completed Tasks}
    \begin{enumerate}
        \item
            Updated GRNN approach data processing technique for HI chapter.
        \item
            Worked on another approach which is HI prognostics approach for RUL for our HI chapter.
        \item
            Discussed on conclusion structure for our survey paper with team.
    \end{enumerate}
\end{subsection}

\begin{subsection}*{Tasks in-progress}
    \begin{enumerate}
        \item
            Working on Third approach HI based prognostics for ML.
        \item
            Working on conclusion for our survey paper with teams.
        \item
            Motivation of our chapter HI.
    \end{enumerate}
\end{subsection}
\end{section}
% Page End here.


% Page start's here.
\newpage
\begin{section}*{Calendar Week: 01 \hfill \date{08 January, 2021}}

\begin{subsection}*{Completed Tasks}
    \begin{enumerate}
        \item
        Working on GRRN Approach for HI chapter.
    \end{enumerate}
\end{subsection}

\begin{subsection}*{Tasks in-progress}
    \begin{enumerate}
        \item
        Working on Motivation and introduction of the chapter HI Estimation. 
    \end{enumerate}
\end{subsection}
\end{section}
% Page End here.

% Page start's here.
\newpage
\begin{section}*{Calendar Week: 52-53 \hfill \date{30 December, 2020}}

\begin{subsection}*{Completed Tasks}
    \begin{enumerate}
        \item
        Working on the structure for Topic HI Estimation.
        \item
            The formal definition for HI.
    \end{enumerate}
\end{subsection}

\begin{subsection}*{Challenges}
    \begin{enumerate}
        \item
            Working on the notation is/was a great headache.
    \end{enumerate}
\end{subsection}

\begin{subsection}*{Tasks in-progress}
    \begin{enumerate}
        \item
        Working on GRNN approaches for Health Index(reference from seed paper \cite{seed_Literature_1.1}) 
    \end{enumerate}
\end{subsection}

\begin{thebibliography}{9}
    \bibitem{seed_Literature_1.1}
    {Islam, Md.Mominul}, 
    “Calculating a health index for power transformers using a subsystem-based GRNN approach",
    \textit{ IEEE Transactions on Power Delivery 33.4}, (2017): 1903-1912.
    \href{ https://ieeexplore.ieee.org/document/8097036}{ https://ieeexplore.ieee.org/document/8097036}
\end{thebibliography}
\end{section}
% Page End here.


% Page start's here.
\newpage
\begin{section}*{Calendar Week: 51 \hfill \date{18 December, 2020}}

\begin{subsection}*{Completed Tasks}
    \begin{enumerate}
        \item
            Create the entire work plan for milestone 2.
        \item
            Working and understanding the time series.
    \end{enumerate}
\end{subsection}

\begin{subsection}*{Tasks in-progress}
    \begin{enumerate}
        \item
            Working on the formal definition for our topic "Heath Index Estimation".
        \item
            Look for the different approaches and methods for estimating the health index.
    \end{enumerate}
\end{subsection}
\end{section}
% Page end here.

% Page start's here.
\newpage
\begin{section}*{Calendar Week: 50 \hfill \date{11 December, 2020}}

\begin{subsection}*{Completed Tasks}
    \begin{enumerate}
        \item
            Deciding on the basic/common introduction section of a combined survey paper(Teamwork).
        \item
            Possible Basic notation that can be used throughout the writing of a survey paper.
    \end{enumerate}
\end{subsection}

\begin{subsection}*{Challenges}
    \begin{enumerate}
        \item
            Finding possible notations and working with the introduced notation by others(Team members).
        \item
            Implementation of pipeline elements.
    \end{enumerate}
\end{subsection}

\begin{subsection}*{Tasks in-progress}
    \begin{enumerate}
        \item
            Searching for different methods which we are going to introduce for our topic "Health index estimation".
        \item
            Working on the implementation of the pipeline element.
        \item
            Estimation of the work plan for the coming weeks.
        \item
            Working on subsection "Motivation and formal definition".
    \end{enumerate}
\end{subsection}

\end{section}
% Page end here.

% Page start's here.
\newpage
\begin{section}*{Calendar Week: 49 \hfill \date{04 December, 2020}}

\begin{subsection}*{Completed Tasks}
    \begin{enumerate}
        \item
            Work with the team to develop adaptive templates and structures for documents.
        \item
            Also, the structure for Git for Milestone 2.
        \item
            Planning the structure of our topic "Health Index Estimation".
    \end{enumerate}
\end{subsection}

\begin{subsection}*{Challenges}
    \begin{enumerate}
        \item
            Lots of ideas, confusion in structuring the subject areas.
        \item
            Some questions about unclear sections like "What does the evaluation setup mean?"
    \end{enumerate}
\end{subsection}

\begin{subsection}*{Tasks in-progress}
    \begin{enumerate}
        \item
            Working on writing a survey on the topic "Health Index Estimation"
        \item
            Working on the facts needed, knowledge extraction from specific survey papers \cite{survey1}\cite{survey2}\cite{survey3}, and seed literature \cite{seed_Literature_1} \cite{seed_Literature_2} for Milestone 2.
    \end{enumerate}
\end{subsection}

\begin{thebibliography}{9}
    \bibitem{survey1}
    {Y. Ran, X. Zhou, P. Lin, Y. Wen, and R. Deng},
    “A survey of predictive maintenance: Systems, purposes and approaches",
    \textit{arXiv preprint arXiv:1912.07383},
    2019,
    \href{https://arxiv.org/pdf/1912.07383.pdf}{https://arxiv.org/pdf/1912.07383.pdf}.
    \bibitem{survey2}
    {W. {Zhang} and D. {Yang} and H. {Wang}},
    "Data-Driven Methods for Predictive Maintenance of Industrial Equipment: A Survey",
    \textit{IEEE Systems Journal},
    vol. 13,
    no. 3,
    pp. 2213-2227,
    2019,
    \href{https://ieeexplore.ieee.org/document/8707108}{https://ieeexplore.ieee.org/document/8707108}.
    \bibitem{survey3}
    {T. P. Carvalho and F. Soares and R. Vita and R. Francisco and Jo{\~a}o P. Basto and Symone G.S. Alcal{\'a}},
    “A systematic literature review of machine learning methods applied to predictive maintenance",
    \textit{Computers \&\ Industrial Engineering},
    vol. 137,
    p. 106024,
    2019.
    \bibitem{seed_Literature_1}
    {Islam, Md.Mominul}, 
    “Calculating a health index for power transformers using a subsystem-based GRNN approach",
    \textit{ IEEE Transactions on Power Delivery 33.4}, (2017): 1903-1912.
    \href{ https://ieeexplore.ieee.org/document/8097036}{ https://ieeexplore.ieee.org/document/8097036}
    \bibitem{seed_Literature_2}
    {Malhotra, Pankaj}, 
    “Multi-sensor prognostics using an unsupervised health index based on LSTM encoder-decoder.",
    \textit{ arXiv preprint arXiv:1608.06154}, 
    (2016).
    \href{ https://arxiv.org/pdf/1608.06154.pdf}{ https://arxiv.org/pdf/1608.06154.pdf}
\end{thebibliography}
\end{section}
% Page End here.


% Page start's here
\newpage
\begin{section}*{Calendar Week: 48 \hfill \date{27 November, 2020}}

\begin{subsection}*{Completed Tasks}
    \begin{enumerate}
        \item Finished reading with survey paper 3 \cite{survey1}.
            This paper aims to provide a systematic literature review covering the major published solutions for PdM techniques based on ML methods.
            \begin{itemize}
            \item Section 3 represents the main steps of the development of an ML model.
            \item Section 4 shows some research-based questions on PdM techniques.
            \item Section 5 gives some insight into data sets that can be used for future research by PdM applications.
            \end{itemize}
    \end{enumerate}
\end{subsection}

\begin{thebibliography}{9}
    \bibitem{survey1}
    {T. P. Carvalho and F. Soares and R. Vita and R. Francisco and Jo{\~a}o P. Basto and Symone G.S. Alcal{\'a}},
    “A systematic literature review of machine learning methods applied to predictive maintenance",
    \textit{Computers \&\ Industrial Engineering},
    vol. 137,
    p. 106024,
    2019.
\end{thebibliography}
\end{section}
% Page end's here


% Page start's here
\newpage

\begin{section}*{Calendar Week: 47 \hfill \date{20 November, 2020}}

\begin{subsection}*{Completed Tasks}
    \begin{enumerate}
        \item
            Set up the Git-Lab environment on my desktop for easy and convenient overflow in the future.
        \item
            Finished reading with the survey paper 2 \cite{survey1}. The general ideas and contributions are as follows:
             \begin{itemize}
             \item General focus was on data-driven methods for PdM.
             \item Review based on a period of five years for industrial applications of PdM from the perspective of DL and ML.
             \end{itemize}
            
    \end{enumerate}
\end{subsection}

\begin{subsection}*{Tasks in-progress}
    \begin{enumerate}
        \item
            Working on survey paper 3 \cite{survey2}. The general ideas and focus of this paper is based on:
            \begin{itemize}
            \item Scientific database which provides a useful Foundation for ML techniques. Which also support new research work in the PdM field 
            \end{itemize}
    \end{enumerate}
\end{subsection}

\begin{thebibliography}{9}
    \bibitem{survey1}
    {W. {Zhang} and D. {Yang} and H. {Wang}},
    “Data-Driven Methods for Predictive Maintenance of Industrial Equipment: A Survey",
    \textit{IEEE Systems Journal},
    vol. 13,
    no. 3,
    pp. 2213-2227,
    2019,
    \href{https://ieeexplore.ieee.org/document/8707108}{https://ieeexplore.ieee.org/document/8707108}.
    \bibitem{survey2}
    {T. P. Carvalho and F. Soares and R. Vita and R. Francisco and Jo{\~a}o P. Basto and Symone G.S. Alcal{\'a}},
    “A systematic literature review of machine learning methods applied to predictive maintenance",
    \textit{Computers \&\ Industrial Engineering},
    vol. 137,
    p. 106024,
    2019.
\end{thebibliography}
\end{section}
% Page end's here


% Page start's here
\newpage

\begin{section}*{Calendar Week: 46 \hfill \date{13 November, 2020}}

\begin{subsection}*{Completed Tasks}
    \begin{enumerate}
        \item
            Finished reading with the survey paper \cite{survey1}. The general ideas and contribution of this paper are as follows:
            \begin{itemize}
                \item PdM system architectures, purposes, and approaches.
                \item A high-level view of the PdM system architectures which, includes OSA-CBM), cloud-enhanced PdM system, and PdM 4.0
            \end{itemize}
    \end{enumerate}
\end{subsection}

\begin{subsection}*{Challenges}
    \begin{enumerate}
        \item
            Understanding through the few topics presented in the survey paper \cite{survey1} was a bit difficult. Real challenges started when Mathematical term/calculation started like for (Cost minimization, run time, reliability, and so on)
    \end{enumerate}
\end{subsection}

\begin{subsection}*{Tasks in-progress}
    \begin{enumerate}
        \item
            Working on the survey paper 2 (Data-Driven Methods for Predictive Maintenance of Industrial Equipment)\cite{survey2}. The main focus of the paper is based on:
            \begin{itemize}
            \item PdM scheme for particular appliances, challenges encounter, approaches. 
            \item Based on six algorithms of ML and DL(deep learning) tries to classify the industrial applications. 
            \end{itemize}
    \end{enumerate}
\end{subsection}

\begin{thebibliography}{9}
    \bibitem{survey1}
    {Y. Ran, X. Zhou, P. Lin, Y. Wen, and R. Deng},
    “A survey of predictive maintenance: Systems, purposes and approaches",
    \textit{arXiv preprint arXiv:1912.07383},
    2019,
    \href{https://arxiv.org/pdf/1912.07383.pdf}{https://arxiv.org/pdf/1912.07383.pdf}.
    \bibitem{survey2}
    {W. {Zhang} and D. {Yang} and H. {Wang}}, “Data-Driven Methods for Predictive Maintenance of Industrial Equipment: A Survey",
    \textit{IEEE Systems Journal},
    vol. 13, 
    no. 3, 
    pp. 2213-2227,
    2019,
    \href{https://ieeexplore.ieee.org/document/8707108}{https://ieeexplore.ieee.org/document/8707108}.
\end{thebibliography}
\end{section}
% Page end's here


% Page start's here
\newpage
\begin{section}*{Calendar Week: 45 \hfill \date{06 November, 2020}}

\begin{subsection}*{Completed Tasks}
    \begin{enumerate}
        \item
            Reading through the material provided for this week on panda for ML4PdM is completed.
        \item
            Working with the team to come up with an idea of “how to structure the “Templates for Minutes and Weekly Reports” report.
    \end{enumerate}
\end{subsection}

\begin{subsection}*{Challenges}
    \begin{enumerate}
        \item
            Coming up with the ideal structure for Templates.
        \item
            Organizing time for Project sessions and working along with the team.
    \end{enumerate}
\end{subsection}

\begin{subsection}*{Tasks in-progress}
    \begin{enumerate}
        \item
            Reading through the survey paper 1 (A Survey of Predictive Maintenance: Systems, Purposes, and Approaches.) \cite{survey1}. General idea of this paper:
            \begin{itemize}
                \item PdM is a maintenance paradigm that performs maintenances after the analytical models predict sudden failures or degradations.
            \end{itemize}
    \end{enumerate}
\end{subsection}

\begin{thebibliography}{9}
    \bibitem{survey1}
    {Y. Ran, X. Zhou, P. Lin, Y. Wen, and R. Deng},
    “A survey of predictive maintenance: Systems, purposes and approaches",
    \textit{arXiv preprint arXiv:1912.07383},
    2019,
    \href{https://arxiv.org/pdf/1912.07383.pdf}{https://arxiv.org/pdf/1912.07383.pdf}.
\end{thebibliography}
\end{section}
% Page end's here

\end{document}