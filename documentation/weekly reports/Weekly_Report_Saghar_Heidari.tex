\documentclass[11pt,a4paper]{article}
\usepackage{graphicx}
\usepackage[utf8]{inputenc}
\usepackage{fancyhdr}
\usepackage{hyperref}
\usepackage[sorting=none, backend=biber]{biblatex}
\usepackage{filecontents}
\begin{filecontents}[overwrite,nosearch]{references.bib}

@inproceedings{DBLP:conf/i2mtc/HelwigPS15,
  author    = {Nikolai Helwig and
               Eliseo Pignanelli and
               Andreas Sch{\"{u}}tze},
  title     = {Condition monitoring of a complex hydraulic system using multivariate
               statistics},
  booktitle = {2015 {IEEE} International Instrumentation and Measurement Technology
               Conference {(I2MTC)} Proceedings, Pisa, Italy, May 11-14, 2015},
  pages     = {210--215},
  publisher = {{IEEE}},
  year      = {2015},
  url       = {https://doi.org/10.1109/I2MTC.2015.7151267},
  doi       = {10.1109/I2MTC.2015.7151267},
  timestamp = {Wed, 16 Oct 2019 14:14:57 +0200},
  biburl    = {https://dblp.org/rec/conf/i2mtc/HelwigPS15.bib},
  bibsource = {dblp computer science bibliography, https://dblp.org}
}
@inproceedings{DBLP:conf/icca/GeramifardXPZL10,
  author    = {Omid Geramifard and
               Jian{-}Xin Xu and
               Chee Khiang Pang and
               Junhong Zhou and
               Xiang Li},
  title     = {Data-driven approaches in health condition monitoring - {A} comparative
               study},
  booktitle = {8th {IEEE} International Conference on Control and Automation, {ICCA}
               2010, Xiamen, China, June 9-11, 2010},
  pages     = {1618--1622},
  publisher = {{IEEE}},
  year      = {2010},
  url       = {https://doi.org/10.1109/ICCA.2010.5524339},
  doi       = {10.1109/ICCA.2010.5524339},
  timestamp = {Wed, 10 Jun 2020 14:27:16 +0200},
  biburl    = {https://dblp.org/rec/conf/icca/GeramifardXPZL10.bib},
  bibsource = {dblp computer science bibliography, https://dblp.org}
}
@article{DBLP:journals/eswa/KimTMC12,
  author    = {Hack{-}Eun Kim and
               Andy Chit Chiow Tan and
               Joseph Mathew and
               Byeong{-}Keun Choi},
  title     = {Bearing fault prognosis based on health state probability estimation},
  journal   = {Expert Syst. Appl.},
  volume    = {39},
  number    = {5},
  pages     = {5200--5213},
  year      = {2012},
  url       = {https://doi.org/10.1016/j.eswa.2011.11.019},
  doi       = {10.1016/j.eswa.2011.11.019},
  timestamp = {Fri, 26 May 2017 22:54:19 +0200},
  biburl    = {https://dblp.org/rec/journals/eswa/KimTMC12.bib},
  bibsource = {dblp computer science bibliography, https://dblp.org}
}
@article{DBLP:journals/tase/CamciC10,
  author    = {Fatih Camci and
               Ratna Babu Chinnam},
  title     = {Health-State Estimation and Prognostics in Machining Processes},
  journal   = {{IEEE} Trans Autom. Sci. Eng.},
  volume    = {7},
  number    = {3},
  pages     = {581--597},
  year      = {2010},
  url       = {https://doi.org/10.1109/TASE.2009.2038170},
  doi       = {10.1109/TASE.2009.2038170},
  timestamp = {Thu, 02 Apr 2020 08:37:02 +0200},
  biburl    = {https://dblp.org/rec/journals/tase/CamciC10.bib},
  bibsource = {dblp computer science bibliography, https://dblp.org}
}
@phdthesis{DBLP:phd/dnb/Kimotho16,
  author    = {James Kuria Kimotho},
  title     = {Development and performance evaluation of prognostic approaches for
               technical systems},
  school    = {University of Paderborn, Germany},
  year      = {2016},
  url       = {https://nbn-resolving.org/urn:nbn:de:hbz:466:2-27129},
  urn       = {urn:nbn:de:hbz:466:2-27129},
  timestamp = {Wed, 12 Feb 2020 16:42:51 +0100},
  biburl    = {https://dblp.org/rec/phd/dnb/Kimotho16.bib},
  bibsource = {dblp computer science bibliography, https://dblp.org}
}
@article{DBLP:journals/corr/abs-1912-07383,
    author    = {Yongyi Ran and
                 Xin Zhou and
                 Pengfeng Lin and
                 Yonggang Wen and
                 Ruilong Deng},
    title     = {A Survey of Predictive Maintenance: Systems, Purposes and Approaches},
    journal   = {CoRR},
    volume    = {abs/1912.07383},
    year      = {2019},
    url       = {http://arxiv.org/abs/1912.07383},
    archivePrefix = {arXiv},
    eprint    = {1912.07383},
    timestamp = {Wed, 01 Jul 2020 09:22:04 +0200},
    biburl    = {https://dblp.org/rec/journals/corr/abs-1912-07383.bib},
    bibsource = {dblp computer science bibliography, https://dblp.org}
}
@article{DBLP:journals/sj/ZhangYW19,
    author    = {Weiting Zhang and
                Dong Yang and
                Hongchao Wang},
    title     = {Data-Driven Methods for Predictive Maintenance of Industrial Equipment:
                {A} Survey},
    journal   = {{IEEE} Syst. J.},
    volume    = {13},
    number    = {3},
    pages     = {2213--2227},
    year      = {2019},
    url       = {https://doi.org/10.1109/JSYST.2019.2905565},
    doi       = {10.1109/JSYST.2019.2905565},
    timestamp = {Fri, 11 Sep 2020 15:01:32 +0200},
    biburl    = {https://dblp.org/rec/journals/sj/ZhangYW19.bib},
    bibsource = {dblp computer science bibliography, https://dblp.org}
}
@article{DBLP:journals/candie/CarvalhoSVFBA19,
    author    = {Thyago Peres Carvalho and
                Fabr{\'{\i}}zzio Alphonsus A. M. N. Soares and
                Roberto Vita and
                Roberto da Piedade Francisco and
                Jo{\~{a}}o P. Basto and
                Symone G. S. Alcal{\'{a}}},
    title     = {A systematic literature review of machine learning methods applied
                to predictive maintenance},
    journal   = {Comput. Ind. Eng.},
    volume    = {137},
    year      = {2019},
    url       = {https://doi.org/10.1016/j.cie.2019.106024},
    doi       = {10.1016/j.cie.2019.106024},
    timestamp = {Mon, 26 Oct 2020 08:24:22 +0100},
    biburl    = {https://dblp.org/rec/journals/candie/CarvalhoSVFBA19.bib},
    bibsource = {dblp computer science bibliography, https://dblp.org}
}
\end{filecontents}
\addbibresource{references.bib}

\title{Project Group:\\
\textbf{Machine Learning for Predictive Maintenance}\\
(Weekly Status Report)}

\author{Supervisors:\\
\textbf{Prof. Dr. Eyke H{\"u}llermeier (eyke@upb.de)}\\
\textbf{Tanja Tornede (tanja.tornede@upb.de)}\\ 
\vspace{10mm}
Name: Saghar Heidari \\
Matriculation Number: 6820583\\
E-mail: sagharh@mail.uni-paderborn.de
}
\date{}

% Definition of \maketitle
\makeatletter
\def \@maketitle{
\begin{center}
\includegraphics[width=10cm, height=3cm]{logo.png}\\
{\vspace{12mm} \@title}\\[4ex]
{\@author}\\[4ex]
\end{center}}
\makeatother

\pagestyle{fancy}
\setlength{\headheight}{14pt}
\rhead{PG ML4PdM}
\lhead{Saghar Heidari}

\begin{document}

\maketitle
\thispagestyle{empty}

\clearpage
\pagenumbering{arabic}
% New page start here.
\newpage
\begin{section}*{Calendar Week: 7 \hfill \date{19 February, 2021}}
	
	\begin{subsection}*{Completed Tasks}
		\begin{enumerate}
                  \item Finished 2nd review of Health Index chapter and prepared a feedback for that chapter.
			\item implemented final changes for health state classification chapter based on the 2nd review : 
			\begin{itemize}
				\item Checking the document for spelling mistakes.
				\item Explaining and adding more details for unclear parts.
				\item Using subsubsection and paragraph instead of textbf.
			\end{itemize}
			\item Had a discussion with Paul, Selami and Vinay about the date format and created a first draft for that.
		\end{enumerate}
	\end{subsection}
	
	\begin{subsection}*{Tasks in-progress}
		\begin{enumerate}
			\item Working with team members to design the data format and write it in latex.
			\item Working with Anurose on the contents of UML diagrams for Health State Classification chapter.
		\end{enumerate}
	\end{subsection}
	
\end{section}
% Page end here.
\newpage
\begin{section}*{Calendar Week: 6 \hfill \date{12 February, 2021}}
      \begin{subsection}*{Completed Tasks}
          \begin{enumerate}
              \item
                   Changed the health state classification chapter based on comments given by @Selami and @Sanjay and updated the alpha version.
          \end{enumerate}
      \end{subsection}
      
      \begin{subsection}*{Tasks in-progress}
          \begin{enumerate}
              \item
                  Reading the Health Index chapter and preparing a feedback for that chapter.
               \item    
                  Updating the health state classification chapter based on second review of @Paul and @Gourav.
            \end{enumerate}
      \end{subsection}
  \end{section}
\newpage
\begin{section}*{Calendar Week: 5 \hfill \date{5 February, 2021}}
      \begin{subsection}*{Completed Tasks}
          \begin{enumerate}
              \item
                  Reviewed the feature extraction  chapter and prepared a feedback for that chapter(added comments).
          \end{enumerate}
      \end{subsection}
      
      \begin{subsection}*{Tasks in-progress}
          \begin{enumerate}
              \item
                  Working on some parts of the health state classification chapter based on comments given by @Selami and @Sanjay.
          \end{enumerate}
      \end{subsection}
  \end{section}
  
\newpage
\begin{section}*{Calendar Week: 4 \hfill \date{29 January, 2021}}


       \begin{subsection}*{Completed Tasks}
             \begin{enumerate}
                  \item 
                  Prepared a feedback for some parts like introduction, motivation,conclusion.
                 
             \end{enumerate}
                 
       \end{subsection}

       \begin{subsection}*{Tasks in-progress}
            \begin{enumerate}
                \item Reading the feature extraction chapter and review it.
            \end{enumerate}
        \end{subsection}
    \end{section}
    


\newpage
\begin{section}*{Calendar Week: 3 \hfill \date{22 January, 2021}}


       \begin{subsection}*{Completed Tasks}
             \begin{enumerate}
                  \item Wrote evaluation section of health state classification chapter.
                 \item Updated the motivation section of health state classification chapter by adding pipeline.
                 \item Updated and correcting the content of chapters.
             \end{enumerate}
                 
       \end{subsection}
\end{section}

\newpage
\begin{section}*{Calendar Week: 2 \hfill \date{15 January, 2021}}
 

       \begin{subsection}*{Completed Tasks}
             \begin{enumerate}
                 \item Completing the remaining part of state of the art approaches(Fuzzy and HMM) and writing knn approach.
                 \item Adding images for each approach to get better understanding.
             \end{enumerate}
                 
       \end{subsection}

       \begin{subsection}*{Tasks in-progress}
             \begin{enumerate}
                   \item  Writing the content of evaluation section of health state classification chapter.
                   \item Writing the motivation section of health state classification .

             \end{enumerate}
       \end{subsection}


\end{section}

\newpage
\begin{section}*{Calendar Week: 1 \hfill \date{8 January, 2021}}
 \begin{refsection}

       \begin{subsection}*{Completed Tasks}
             \begin{enumerate}
                 \item wrote the content of dataset,namely,Condition monitoring of hydraulic systems Data Set \cite{DBLP:conf/i2mtc/HelwigPS15}and Bearing Fault Dataset.
                 \item wrote first draft of state of art approaches (Fuzzy and HMM ).
             \end{enumerate}
                 
       \end{subsection}

       \begin{subsection}*{Tasks in-progress}
             \begin{enumerate}
                   \item  Writing remaining state of art approach and completing it.

             \end{enumerate}
       \end{subsection}

       \printbibliography
 \end{refsection}
\end{section}

\newpage
\begin{section}*{Calendar Week: 52-53 \hfill \date{30 December, 2020}}
 \begin{refsection}

       \begin{subsection}*{Completed Tasks}
             \begin{enumerate}
                 \item Working with Anurose to find related Datasets,namely, Condition monitoring of hydraulic systems Data Set \cite{DBLP:conf/i2mtc/HelwigPS15}and Bearing Fault Dataset , Bearing Data Set from NASA. 
                 \item Finding papers related to state of art approaches (KNN, Neuro-Fuzzy) of health state classification and understanding most important part of each methods.
                 \item Discussed with Anurose for correcting Formal definition of health state classification.
             \end{enumerate}
                 
       \end{subsection}
       
       \begin{subsection}*{Challenges}
		\begin{enumerate}
			\item Finding related data-sets were difficult and took more time. Most of available data set address fault detection problem(binary target classes, faulty/healthy).For instance, CNC Mill Tool Wear dataset, Air pressure system failures in Scania trucks dataset.
		\end{enumerate}
	\end{subsection}

       \begin{subsection}*{Tasks in-progress}
             \begin{enumerate}
                   \item  Searching more state-of-the-art approaches papers. (HMMs, KNN, Neuro-Fuzzy).
                   \item Writing the content of dataset part.

             \end{enumerate}
       \end{subsection}

       \printbibliography
 \end{refsection}
\end{section}

\newpage
\begin{section}*{Calendar Week: 51 \hfill \date{18 December, 2020}}
 \begin{refsection}

       \begin{subsection}*{Completed Tasks}
             \begin{enumerate}
                 \item Writing Introduction of the Topic study.
                 \item Discussed with Anurose and created pipeline structure for health state classification.
                 \item Discussed with Anurose about formal definition of health state classification and with other team members for Formal definition of time series.
                 \item Discussed with Anurose about work plan for the second milestone.
             \end{enumerate}
                 
       \end{subsection}
       
       \begin{subsection}*{Challenges}
		\begin{enumerate}
			\item Creating formal definition and pipeline structure took more time.
		\end{enumerate}
	\end{subsection}

       \begin{subsection}*{Tasks in-progress}
             \begin{enumerate}
                   \item Understanding of Data-driven approaches in health condition monitoring to find the related data sets. 
                   \item  Working with Anurose on list of data-sets and state-of-the-art approaches.
             \end{enumerate}
       \end{subsection}

       \printbibliography
 \end{refsection}
\end{section}

\newpage
\begin{section}*{Calendar Week: 50 \hfill \date{11 December, 2020}}
 \begin{refsection}

       \begin{subsection}*{Completed Tasks}
             \begin{enumerate}
                 \item  Created minutes for Wednesday's meeting (9 December 2020) 
                 \item Finished reading survey "Health-State Estimation and Prognostics in Machining Processes" \cite{DBLP:journals/tase/CamciC10} which is focused on two approaches (regular HMM and hierarchical HMM) for health-state and calculating remaining- useful-life (RUL) estimation using Monte Carlo simulation. These approaches are implemented using example of Drill bits.
                 \item Assigning sections and subsections with Anurose.
             \end{enumerate}
                 
       \end{subsection}

       \begin{subsection}*{Tasks in-progress}
             \begin{enumerate}
                   \item Searching and reading much more papers such as Reading 
                        survey "Bearing fault prognosis based on health state probability estimation" \cite{DBLP:journals/eswa/KimTMC12} and "Data-driven approaches in health condition monitoring - {A} comparative"\cite {DBLP:conf/icca/GeramifardXPZL10} and so on.
                        \item Working on motivation content of health state classification
                        \item Working on Introduction of main Document of PG.
                        \item Joined with Anurose to creat formal definition of health state classification.
             \end{enumerate}
       \end{subsection}

       \printbibliography
 \end{refsection}
\end{section}

\newpage
\begin{section}*{Calendar Week: 49 \hfill \date{04 December, 2020}}
 \begin{refsection}

       \begin{subsection}*{Completed Tasks}
             \begin{enumerate}
                   \item Discussing and working with team members on tasks such as the report template for topic studies and decide on structures for document.
                    \item Reading the dissertation "Development and performance evaluation of prognostic approaches for technical systems"\cite{DBLP:phd/dnb/Kimotho16} which discussed the following points as follows:
                         \begin{itemize}
                             \item Presenting prognostics and health management (PHM) as a maintenance strategy for detecting faults and predicting RUL and describing essential elements of it. 
                            \item Giving an overview of fault diagnostic methods which identify fault type and location as a main cause of unusual system behavior with two types model-based and data-driven.
                            \item Explaining classification of Data-driven approaches for PHM in Statistical approach and Machine learning approach, in which pattern recognition based on supervised and unsupervised ML methods is used for detecting faults and their applications are discussed.
                            \item Giving an overview of fault prognostic methods which is focused on estimating the remaining useful lifetime (RUL) of a system namely, Reliability based methods using failure times of similar units, Model-driven approach employing mathematical models of system, and Data-driven approach utilizing condition monitoring data to learn degradation behavior of a system and the evaluation of each method.
                            \item Introducing prognostics approaches depending on the type of condition monitoring data available such as health state estimation with help of ML algorithms (SVM, RF, PNN), estimating RUL and so on.
                        \end{itemize}
             \end{enumerate}
       \end{subsection}

       \begin{subsection}*{Tasks in-progress}
             \begin{enumerate}
                   \item Working on the seed literature \cite{DBLP:journals/tase/CamciC10} and \cite{DBLP:journals/eswa/KimTMC12} to get familiar with the topic of health state classification.
                   \item Deciding (jointly with team member Anurose) about elements of pipeline, formal definition, sections and subsections of health state classification.
             \end{enumerate}
       \end{subsection}

       \printbibliography
 \end{refsection}
\end{section}

\newpage
\begin{section}*{Calendar Week: 48 \hfill \date{27 November, 2020}}
 \begin{refsection}

       \begin{subsection}*{Completed Tasks}
             \begin{enumerate}
                   \item
                         Finished reading Survey "Data-Driven Methods for Predictive Maintenance of Industrial Equipment" \cite{DBLP:journals/sj/ZhangYW19}.
                        In this paper, following topics are discussed:
                         \begin{itemize}
                             \item Reviewing maintenance methods in which PdM leads to more promising results.
                            \item Explaining PdM approaches,namely, Knowledge-based prognosis, model-based prognosis
                            and data driven prognosis where for smart manufacturing and industrial equipment, Data-Driven Models show a better performance in terms of maintenance cost reduction and reducing equipment downtime.
                            \item Implementation of Data-Driven PdM system in four stages(operational assessment, data acquisition, feature engineering and modeling) and proposing the PdM system using the example of automatic washing equipment and its challenges.
                            \item Introducing six ML and DL algorithms and their evaluation based on five proposed metrics of signal type, application scenario, target, accuracy, and data sets, among which accuracy was discussed in detail as the most important metric.
                        \end{itemize}
             \end{enumerate}
       \end{subsection}

       \begin{subsection}*{Tasks in-progress}
             \begin{enumerate}
                   \item
                     Reading the dissertation "Development and performance evaluation of prognostic approaches for technical systems" \cite{DBLP:phd/dnb/Kimotho16}.
             \end{enumerate}
       \end{subsection}

       \printbibliography
 \end{refsection}
\end{section}
% Page End here.
\newpage
\begin{section}*{Calendar Week: 47 \hfill \date{20 November, 2020}}
 \begin{refsection}

       \begin{subsection}*{Completed Tasks}
             \begin{enumerate}
                   \item
                          Finished reading first Survey \cite{DBLP:journals/corr/abs-1912-07383}  which is focused on three types of maintenance strategies and their differences. It also introduces Technologies and system architectures and the Ways through which they contribute to PdM.
                         \begin{itemize}
                            \item In final part, it gives an overview of DL based Methods for fault diagnosis and prognosis in PdM, namely, Auto encoder, CNN, RNN and so on.
                            \par The survey also describes the features,advantages, disadvantages and different applications for each DL-based approach to select best DL method for particular PdM applications.
                        \end{itemize}
             \end{enumerate}
       \end{subsection}

       \begin{subsection}*{Challenges}
             \begin{enumerate}
                   \item
                         Understanding complex structures of some DL based approaches.
             \end{enumerate}
       \end{subsection}

       \begin{subsection}*{Tasks in-progress}
             \begin{enumerate}
                   \item
                        Reading survey "Data-Driven Methods for Predictive Maintenance of Industrial Equipment" \cite{DBLP:journals/sj/ZhangYW19}
             \end{enumerate}
       \end{subsection}

       \printbibliography
 \end{refsection}
\end{section}
% Page End here.
%If u want to add more pages? --> copy the below code and paste it here.
\newpage
\begin{section}*{Calendar Week: 46 \hfill \date{13 November, 2020}}
 \begin{refsection}

       \begin{subsection}*{Completed Tasks}
             \begin{enumerate}
                   \item
                         Finished reading Survey 3 \cite{DBLP:journals/candie/CarvalhoSVFBA19}. which its highlighted points are as follows:
                         \begin{itemize}
                             \item Systematic literature review to recognize and categorize relevant literature to PdM using ML methods.
                            \item The most used ML methods for PdM in Literature, namely, Random Forest , ANN, SVM, K-Mean (which are referred as Traditional ML based approaches in Survey 1).
                        \end{itemize}
             \end{enumerate}
       \end{subsection}

       \begin{subsection}*{Challenges}
             \begin{enumerate}
                   \item
                        Some Parts of Survey 1 were complicated and for a better understanding of basic concepts, I started with reading Survey 3.
             \end{enumerate}
       \end{subsection}

       \begin{subsection}*{Tasks in-progress}
             \begin{enumerate}
                   \item
                         Reading the remaining part of first Survey : DL based Approaches \cite{DBLP:journals/corr/abs-1912-07383}.
             \end{enumerate}
       \end{subsection}

       \printbibliography
 \end{refsection}
\end{section}
% Page End here.

% New page start here.
\newpage
\begin{section}*{Calendar Week: 45 \hfill \date{06 November, 2020}}
 \begin{refsection}

       \begin{subsection}*{Completed Tasks}
             \begin{enumerate}
                   \item
                         Joined and followed Christopher and Paul by setting up JIRA and GitLab and getting familiar with the process and platforms.
                   \item
                        Designed Weekly Report Template in Word and converted in Latex format (completed and uploaded by Sanjay).
             \end{enumerate}
       \end{subsection}

       \begin{subsection}*{Challenges}
             \begin{enumerate}
                   \item
                         Getting familiar with basic concepts in PdM.
             \end{enumerate}
       \end{subsection}

       \begin{subsection}*{Tasks in-progress}
             \begin{enumerate}
                   \item
                        Reading relevant literature (First Survey)\cite{DBLP:journals/corr/abs-1912-07383}.
                        \begin{itemize}
                            \item Existing  system architectures of PdM and their structures and standards(OSA-CBM,Cloud-Enhanced PdM systems,Pdm 4.0).
                            \item Organizing the possible approaches in PdM (knowledge based, traditional ML based,DL based approaches).
                            \item Think about advantages and disadvantages 
                             of knowledge based approach.
                        \end{itemize}
             \end{enumerate}
       \end{subsection}

       \printbibliography
 \end{refsection}
\end{section}
% Page end here.

\end{document}
